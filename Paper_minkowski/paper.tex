%\documentclass[useAMS, usenatbib,usegraphicx,letter]{mn2e}
%\documentclass[11pt]{article}
\documentclass[reprint,aps,prd,superscriptaddress,showkeys,showpacs]{revtex4-1}
\usepackage{epsfig,amsmath,natbib}

\usepackage{aas_macros}
\usepackage{amssymb}
\usepackage{amsmath}
\usepackage{dsfont}
\usepackage{hyperref}
\usepackage{color}
\usepackage{pbox}

\hypersetup{
	colorlinks=false,
	citecolor=green
}
% \usepackage{graphicx}
% \usepackage{epstopdf}
% \usepackage{natbib}

\begin{document}

\title{Moments and Minkowski Functionals constrain cosmology: applying a weak lensing emulator to CFHTLenS data}

\author{Andrea Petri}
\email{apetri@phys.columbia.edu}
\affiliation{Department of Physics, Columbia University, New York, NY 10027, USA}
\affiliation{Physics Department, Brookhaven National Laboratory, Upton, NY 11973, USA}

\author{Jia Liu}
\affiliation{Department of Astronomy, Columbia University, New York, NY 10027, USA}

\author{Zolt\'an Haiman}
\affiliation{Department of Astronomy, Columbia University, New York, NY 10027, USA}

\author{Morgan May}
\affiliation{Physics Department, Brookhaven National Laboratory, Upton, NY 11973, USA}

\author{Lam Hui}
\affiliation{Department of Physics, Columbia University, New York, NY 10027, USA}

\author{Jan M. Kratochvil}
\affiliation{Astrophysics and Cosmology Research Unit, University of KwaZulu-Natal, Westville, Durban 4000, South Africa}

\date{\today}

\label{firstpage}

%MM edited wording

\begin{abstract}
Weak gravitational lensing is a potentially powerful probe for cosmology, and in particular for the Dark Energy (DE) equation of state $w$; since typical weak lensing fields are non--Gaussian in nature, quadratic statistics, such as angular power spectra, can miss significant cosmological information contained in survey data. In this work we examine constraints on the parameter triplet $(\Omega_m,w,\sigma_8)$ derived from weak lensing Minkowski Functionals and Moments, using publicly available data from the 154\,deg$^2$ CFHTLenS survey. We utilize a new suite of ray--tracing N-body simulations spanning 91 points in the $(\Omega_m,w,\sigma_8)$ parameter space, replicating the galaxy sky positions, redshift and shape noise in the CFHTLenS catalogs. We then build an emulator that interpolates the simulated feature spaces, and use it to compute the parameter likelihood, from which we derive the data constraints. We find that the dimensionality reduction technique Principal Component Analysis helps in stabilizing the constraints with respect to the number of bins used to construct the convergence statistics. Using our full set of statistical descriptors, we constrain $\Sigma_8=\sigma_8(\Omega_m/0.27)^{0.55}$ to a value of $0.75\pm0.04$ at the $1\sigma$ significance level, well in agreement with the values already quoted in the literature. We find that, while the Minkowski Functionals constraints on the $(\Omega_m,\sigma_8)$ doublet suffer from a bias, the Moments provide a tight unbiased bound on the parameters, with a substantial contribution coming from quartic moments of gradients.  

\end{abstract}

%AP: Re-worded the abstract

\keywords{Weak Gravitational Lensing --- Data analysis --- Methods: analytical,numerical,statistical}
\pacs{98.80.-k, 95.36.+x, 95.30.Sf, 98.62.Sb}

\maketitle


%%%%%%%%%%%%%%%%%%%%%%%%%% INTRO %%%%%%%%%%%%%%%%%%%%%%%%%%%%%%%%%%%%%%%%%%%%%%%%%%%%%%%%

\section{Introduction}
%

Weak gravitational lensing is emerging as a promising technique to constrain cosmology. Techniques have been developed to construct cosmic shear fields with shape measurements in large galaxy catalogues; although the shear two--point function (2PCF) serves as a main cosmological probe (see \citep{CFHTKilbinger} for example), alternative statistics have been shown to increase the amount of cosmological information one can extract from weak lensing fields. Among these, higher moments (\citep{moments1,moments2,moments3,moments4,moments5,moments6}), three--point functions (\citep{3pcf1,3pcf2}), bispectra (\citep{bispectrum1,bispectrum2,bispectrum3,bispectrum4}), peak counts (\citep{peaks1,peaks2,peaks3,peaks4,peaks5,peaks6}) and Minkowski Functionals (\citep{MinkJan,Petri2013}) have shown to improve cosmological constraints in weak lensing analyses.  
In this work we use the publicly available CFHTLenS data, consisting of a catalog of $\approx$4.2 million galaxies, combined with a suite of ray-tracing simulations in 91 different cosmological models to constrain the cosmological parameters $\Omega_m$,$\sigma_8$ and the DE equation of state $w$. The paper is organized as follows; we first give an overview of the CFHTLenS catalogs, summarizing the data reduction techniques adopted. Next, we give a description of our simulation pipeline, including the procedure used to sample the parameter space and outline the ray--tracing algorithm. We call the weak lensing statistical observables - the Power Spectrum, Minkowski Functionals and Moments - "descriptors" or "features" interchangeably throughout the paper. We discuss the calculation of the statistical descriptors, the dimensional reduction method, and the statistical inference framework used. We then outline our results, and the the cosmological parameter constraints obtained. To conclude, we discuss our findings and comment on possible future extensions of this analysis.  

% "descriptors" or "features" are used interchangeably throughout the paper. It may be sufficient to state this here, but it might be clearer to use only "descriptors" Morgan 2/16/2015

%%%%%%%%%%%%%%%%%%%%%%%%%% DATA AND SIMULATIONS %%%%%%%%%%%%%%%%%%%%%%%%%%%%%%%%%%%%%%%%%%%%%%%%%%

\section{Data and simulations}

\subsection{CFHTLenS data reduction}
\label{cfhtdatareduction}
%
The CFHTLenS survey covers four sky patches of 64, 23, 44 and 23 deg$^2$ area, for a total of 154 deg$^2$; the publicly released data consist of a galaxy catalog created using SExtractor \citep{SExtractor}, a photometric redshift estimated with a Bayesian photometric redshift code \citep{PhotoCode} and galaxy shape measurements made using \textit{lensfit} \citep{cfht1,cfht2}. The cosmological parameter inferences have been obtained in \citep{CFHTKilbinger} using the 2PCF; additionally a number of authors investigated the CFHTLenS constraining power using statistics that go beyond the usual quadratic ones: \citep{CFHTFu} investigated the effect of using the skewness as an additional probe for cosmology, while \citep{CFHTMasato} investigated CFHT cosmological constraints and systematic errors using the Minkowski Functionals. We apply the following cuts to the galaxy catalog: mask$<1$, redshift $0.2 < z < 1.3$, fitclass = 0 (which requires the object to be a galaxy) and weight $w>0$ (with larger $w$ indicating smaller shear measurement uncertainty). Applying these cuts leaves us 4.2$\times10^6$ galaxies, 124.7 deg$^2$ sky coverage, and average galaxy density $n_{gal} \approx 9.3\,\mathrm{arcmin}^{-2}$. The catalog is further reduced by $\sim25\%$ when one rejects fields with non--negligible star--galaxy correlations. These spurious correlations are likely due to imperfect PSF removal, and do not contain cosmological signal. The CFHTLenS galaxy catalog provides us with the sky position $\pmb{\theta}$, the redshift $z(\pmb{\theta})$ and ellipticity $\mathbf{e}(\pmb{\theta})$ of each galaxy, as well as the individual weight factors $w(\pmb{\theta})$ and additive and multiplicative ellipticity corrections $c(\pmb{\theta}), m(\pmb{\theta})$. Because the CFHTLenS fields are irregularly shaped, we first divide them into 13 squares (subfields) to match the shape and $\approx12$ deg$^2$ size of our simulated maps; these square subfield maps are pixelized according to a Gaussian gridding procedure

%AP: added references and fixed typo. Is there a problem with the galaxy density I reported? (I took it from Jia's paper) 
\begin{equation}
\bar{\mathbf{e}}(\pmb{\theta}) = \frac{\sum_{i=1}^{N_s} W(\vert\pmb{\theta}-\pmb{\theta}_i\vert)w(\pmb{\theta}_i)[\mathbf{e}^{obs}(\pmb{\theta}_i)-c(\pmb{\theta}_i)]}{\sum_{i=1}^{N_s}W(\vert\pmb{\theta}-\pmb{\theta}_i\vert)w(\pmb{\theta}_i)[1+m(\pmb{\theta})]}
\end{equation} 
\begin{equation}
\label{gausskernel}
W_{\theta_G}(\pmb{\theta}) = \frac{1}{2\pi\theta_G^2}\exp{\left(-\frac{\pmb{\theta}^2}{2\theta_G^2}\right)}
\end{equation}
%
where the smoothing scale $\theta_G$ has been fixed at 1.0\,arcmin (and occasionally varied to 1.8 and 3.5\,arcmin for isolated tests); the multiplicative and additive corrections $m,c$ relate the observed and true ellipticities of the galaxies in the catalog
\begin{equation}
\mathbf{e}^{obs} = (1+m)\mathbf{e}^{true} + c
\end{equation}
%
Using the ellipticity grid $\bar{\mathbf{e}}(\pmb{\theta})$ as an estimator for the cosmic shear $\gamma^{1,2}(\pmb{\theta})$, we can perform a non--local Kaiser--Squires inversion \citep{KS} to recover the convergence $\kappa(\pmb{\theta})$ from the $E$--mode of the shear field
%
\begin{equation}
\kappa(\mathbf{l}) = \left(\frac{l_1^2-l_2^2}{l_1^2+l_2^2}\right)\gamma^1(\mathbf{l}) + 2\frac{l_1l_2}{l_1^2+l_2^2}\gamma^2(\mathbf{l})
\end{equation}
%
The CFHTLenS catalogs contain masked regions ( rejected fields, regions around bright stars etc.). We first create gridded maps of the same size and resolution as the $\kappa$ maps, but with each pixel containing the number of galaxies ($n_{gal}$) falling within its window. We then smooth this galaxy surface density map with the same Gaussian window function as equation (\ref{gausskernel}) and we remove regions where $n_{gal} < 5 \,\mathrm{arcmin}^{−2}$ (see \citep{CFHTMasato}). Regions with low galaxy number density can induce large errors in the cosmological parameter inferences, hence they need to be removed. For a more in--depth description of our data reduction procedure, we refer the reader to \citep{Companion}. 

%AP: throughout --> in--depth

\subsection{Simulation design}
In this section we give a description of the method we used to sample the parameter space in our simulation effort. We wish to investigate the non--linear dependence of cosmological probes (in this work Minkowski Functionals and Moments of the $\kappa$ field) on the parameter triplet $\mathbf{p}=(\Omega_m,w,\sigma_8)$, while keeping the other relevant parameters $(h,\Omega_b,n_s)$ fixed to the values (0.7, 0.046, 0.96). We sampled the $D$--dimensional ($D=3$ in this case) parameter space using an irregularly spaced grid, designed with a method similar to \citep{coyote2}; the irregular grid design is more efficient, given the available computing resources: a regular parameter grid with the same average spacing between models would require a prohibitively large number of samples. We limit the parameter sampling in a box of corners $\Omega_m\in[0.07,1],\,w\in[-3.0,0],\,\sigma_8\in[0.1,1.5]$ and we map this sampling box $\Pi$ into an hypercube of unit side; we want to contruct an irregularly spaced grid consisting of $N$ points $\mathbf{x}_i\in[0,1]^D$. Let a \textit{design} $\mathcal{D}$ be the set of this irregularly spaced $N$ points: we wish to find an optimal design, in which the points are spread as uniformly as possible inside the box. Following \citep{coyote2}, we choose our optimal design as the minimum of the cost function

%AP: Justified irregular parameter sampling

\begin{equation}
\label{costfunction}
d(\mathcal{D}) = \frac{2D^{1/2}}{N(N-1)}\sum_{i<j}^N\frac{1}{\vert\mathbf{x}_i-\mathbf{x}_j\vert}
\end{equation} 
%
This problem is mathematically equivalent to the minimization of the Coulomb potential energy of $N$ unit charges in a unit box, which will make sure that the charges are as evenly spread as possible throughout the confining volume. Finding the optimal design $\mathcal{D}_m$ that minimized (\ref{costfunction}) can be computationally very demanding, and hence we decided to use a simplified approach that, although approximate, serves our purposes for the grid design. We use an iterative procedure:
\begin{enumerate}
\item We start from the diagonal design $\mathcal{D}_0$: 

$x_i^d\equiv i/(N-1)$
\item We shuffle the coordinates of the particles in each dimension independently $x_i^d = \mathcal{P}_d\left(\frac{1}{N-1},\frac{2}{N-1},...,1\right)$ where $\mathcal{P}_1,...,\mathcal{P}_D$ are random independent permutations of $(1,2,...,N)$
\item We pick a random particle pair $(i,j)$ and a random coordinate $d\in\{1,...,D\}$ and swap $x_i^d\leftrightarrow x_j^d$
\item We compute the new cost function, if this is less than the previous step, we keep the exchange, otherwise we revert the coordinate swap
\item We repeat steps 3 and 4 until the relative cost function change is less than a chosen accuracy parameter $\epsilon$ 
\end{enumerate}
%
We found that for $N=91$ grid points, order of $10^5$ iterations are sufficient to reach an accuracy of $\epsilon\sim10^{-4}$; once the optimal design $\mathcal{D}_m$ has been found, we can invert the mapping $\Pi\rightarrow[0,1]^3$ to find our simulation parameter sampling $\mathbf{p}_s$, which we show in Table \ref{designtable} and Figure \ref{designfig}.
%
\begin{table*}
\begin{tabular}{c|ccc||c|ccc||c|ccc||c|ccc}
$N$ & $\Omega_m$ & $w$ & $\sigma_8$ & $N$ & $\Omega_m$ & $w$ & $\sigma_8$ & $N$ & $\Omega_m$ & $w$ & $\sigma_8$ & $N$ & $\Omega_m$ & $w$ & $\sigma_8$ \\ \hline
1 & 0.136 & -2.484 & 1.034 & 26 & 0.380 & -2.424 & 0.199 & 51 & 0.615 & -1.668 & 0.185 & 76 & 0.849 & -0.183 & 0.821 \\
2 & 0.145 & -2.211 & 1.303 & 27 & 0.389 & -0.939 & 0.454 & 52 & 0.624 & -2.757 & 0.327 & 77 & 0.859 & -1.182 & 1.415 \\
3 & 0.155 & -0.393 & 0.652 & 28 & 0.399 & -1.938 & 1.500 & 53 & 0.634 & -1.575 & 0.976 & 78 & 0.869 & -2.031 & 0.227 \\
4 & 0.164 & -2.181 & 0.313 & 29 & 0.409 & -2.940 & 0.737 & 54 & 0.643 & -2.454 & 1.444 & 79 & 0.878 & -2.697 & 0.524 \\
5 & 0.173 & -0.423 & 1.231 & 30 & 0.418 & -1.758 & 0.383 & 55 & 0.652 & -1.029 & 1.458 & 80 & 0.887 & -0.363 & 0.439 \\
6 & 0.183 & -0.909 & 0.269 & 31 & 0.427 & -2.910 & 0.411 & 56 & 0.661 & -0.486 & 0.892 & 81 & 0.897 & -0.999 & 0.468 \\
7 & 0.192 & -1.605 & 1.401 & 32 & 0.436 & -0.060 & 0.878 & 57 & 0.671 & -2.364 & 0.793 & 82 & 0.906 & -1.698 & 1.273 \\
8 & 0.201 & -2.787 & 0.807 & 33 & 0.446 & -1.212 & 1.486 & 58 & 0.681 & -2.970 & 0.610 & 83 & 0.915 & -2.544 & 1.175 \\
9 & 0.211 & -0.333 & 0.341 & 34 & 0.455 & -2.637 & 1.373 & 59 & 0.690 & -1.332 & 0.482 & 84 & 0.925 & -0.636 & 1.259 \\
10 & 0.221 & -1.485 & 0.666 & 35 & 0.464 & -2.121 & 0.906 & 60 & 0.700 & -0.273 & 0.283 & 85 & 0.943 & -2.394 & 0.835 \\
11 & 0.239 & -1.848 & 0.962 & 36 & 0.474 & -1.302 & 0.114 & 61 & 0.709 & -2.061 & 0.425 & 86 & 0.953 & -1.545 & 0.355 \\
12 & 0.249 & -2.727 & 0.369 & 37 & 0.483 & -1.515 & 0.680 & 62 & 0.718 & -1.728 & 1.472 & 87 & 0.963 & -2.151 & 0.510 \\
13 & 0.258 & -1.395 & 0.241 & 38 & 0.493 & -0.243 & 0.297 & 63 & 0.728 & -0.120 & 0.596 & 88 & 0.972 & -0.666 & 0.694 \\
14 & 0.267 & -2.667 & 1.317 & 39 & 0.502 & -1.152 & 1.189 & 64 & 0.737 & -2.847 & 1.203 & 89 & 0.981 & -1.242 & 1.048 \\
15 & 0.276 & -0.849 & 1.429 & 40 & 0.512 & -0.819 & 0.849 & 65 & 0.746 & -0.090 & 1.118 & 90 & 0.991 & -1.908 & 1.020 \\
16 & 0.286 & -1.272 & 1.104 & 41 & 0.521 & -2.334 & 0.538 & 66 & 0.755 & -0.456 & 1.359 & 91 & 1.000 & -1.425 & 0.708 \\
17 & 0.295 & -1.878 & 0.100 & 42 & 0.530 & 0.000 & 0.624 & 67 & 0.765 & -2.091 & 1.076 & -- & -- & -- & -- \\
18 & 0.305 & -0.879 & 0.765 & 43 & 0.540 & -0.030 & 1.161 & 68 & 0.775 & -1.122 & 1.132 & -- & -- & -- & -- \\
19 & 0.315 & -2.241 & 0.638 & 44 & 0.549 & -1.818 & 1.287 & 69 & 0.784 & -1.062 & 0.779 & -- & -- & -- & -- \\
20 & 0.324 & -2.001 & 1.217 & 45 & 0.558 & -2.577 & 1.146 & 70 & 0.794 & -1.365 & 0.156 & -- & -- & -- & -- \\
21 & 0.333 & -0.213 & 0.552 & 46 & 0.568 & -0.516 & 1.331 & 71 & 0.803 & -2.607 & 0.255 & -- & -- & -- & -- \\
22 & 0.342 & -2.817 & 1.062 & 47 & 0.577 & -3.000 & 0.948 & 72 & 0.812 & -1.788 & 0.722 & -- & -- & -- & -- \\
23 & 0.352 & -0.576 & 1.090 & 48 & 0.587 & -2.304 & 0.128 & 73 & 0.821 & -2.880 & 0.863 & -- & -- & -- & -- \\
24 & 0.361 & -0.606 & 0.171 & 49 & 0.596 & -0.696 & 0.496 & 74 & 0.831 & -0.759 & 0.213 & -- & -- & -- & -- \\
25 & 0.370 & -0.303 & 1.345 & 50 & 0.606 & -0.789 & 0.142 & 75 & 0.840 & -2.274 & 1.387 & -- & -- & -- & -- \\
\end{tabular}
\caption{List of the CFHTemu1 grid points in parameter space}
\label{designtable}
\end{table*}
%
\begin{figure*}
\begin{center}
\includegraphics[scale=0.4]{Figures/design.eps}
\caption{$(\Omega_m,w)$ and $(\Omega_m,\sigma_8)$ projections of our the simulation design; the blue points correspond to the CFHTemu1 simulation set, which consists of one $N$--body simulation per point, while the red point corresponds to the CFHTcov simulation set, which is based on 50 independent $N$--body simulations}
\label{designfig}
\end{center}
\end{figure*}
%
For each parameter point on the grid $\mathbf{p}_s$ we run one $N$--body simulation and perform ray tracing through it, as described in \S~\ref{raysim}, to simulate CFHTLenS shear catalogs; this set of simulations will be called CFHTemu1 throughout the rest of this work. Additionally, we run 50 independent $N$--body simulations with a \textit{fiducial} parameter choice $\mathbf{p}_0=(0.26,-1.0,0.8)$ for the purpose of measuring accurately the covariance matrices which will be used for the parameter inferences in \S-\ref{cosmostats}; this set of simulations will be called CFHTcov throughout the rest of this work.    

\subsection{Ray Tracing Simulations}
\label{raysim}
The goal of this section is to give an outline of our simulation pipeline; the fluctuations in the matter density field between a source at redshift $z$ and an observer located on Earth will cause small deflections to the trajectories of light rays traveling from the source to the observer. The fluctuations in the matter field are described by the gravitational potential $\Phi(\mathbf{x},z)=\Phi(\mathbf{x}_\perp,\chi(z))$, where we can replace the physical coordinates $\mathrm{x}$ with the comoving distance from the observer $\chi$ and two transverse coordinates $\mathbf{x}_\perp=\chi\pmb{\beta}$ using the flat sky approximation. Here $\pmb{\beta}$ refers to the angular coordinate on the sky of a physical point $\mathbf{x}$, as seen from the observer. We estimate the dark matter gravitational potential running $N$--body simulations (with $N=512^3$) with the public code Gadget2 \citep{Gadget2}, using a comoving box size of $240h^{-1}$Mpc. Using a similar procedure as in \citep{RayTracingJain,RayTracingHartlap}, the equation that governs the light ray deflections can be written in the form
\begin{equation}
\label{raytrajectory}
\frac{d^2\mathbf{x}(\chi)}{d\chi^2} = -\frac{2}{c^2}\nabla_{\mathbf{x}_\perp}\Phi(\mathbf{x}_\perp(\chi),\chi)
\end{equation}
%
where $\mathbf{x}(\chi)$ is the trajectory of a single light ray. Suppose that a light ray reaches the observer at an angular position $\pmb{\theta}$ on the sky: we want to know where this light ray originated, knowing it comes from a redshift $z_s$. To answer this question we need to integrate equation (\ref{raytrajectory}) with the initial conditions $\pmb{\beta}(0;\pmb{\theta})=\pmb{\theta}$, $\dot{\pmb{\beta}}(0;\pmb{\theta})=0$ up to a distance $w_s=w(z_s)$ to obtain the source angular position $\pmb{\beta}(w_s;\pmb{\theta})$; for the light ray trajectory solver, based on equation (\ref{raytrajectory}), we use our proprietary implementation Inspector Gadget. Once we know the details of the light ray trajectories, we can easily infer the interesting weak lensing  quantities by taking the angular derivatives of the ray deflections $A(\chi_s;\pmb{\theta}) = \partial \pmb{\beta}(\chi_s;\pmb{\theta})/\partial\pmb{\theta}$ and performing the usual spin decomposition to infer the convergence $\kappa$ and the shear components $(\gamma^1,\gamma^2)$
%
\begin{equation}
A(\chi_s;\pmb{\theta}) = (1-\kappa(\chi_s;\pmb{\theta}))\pmb{I} - \gamma^1(\chi_s;\pmb{\theta})\sigma^3 - \gamma^2(\chi_s;\pmb{\theta})\sigma^1
\end{equation}  
%
where $\pmb{I}$ is the $2\times2$ identity and $\sigma^{1,3}$ are the first and third Pauli matrices. $\kappa$ is related to the source apparent magnification, while $(\gamma^1,\gamma^2)$ are related to the source apparent ellipticity, as seen by the observer. Given a source with intrinsic ellipticity $\mathbf{e}_s=e^1_s + ie^2_s$, its observed ellipticity as seen by an observer will be modified by the cosmic shear $\pmb{\gamma}=\gamma^1 + i\gamma^2$ following
%
\begin{equation}
\mathbf{e} = 
\begin{cases}
\frac{\mathbf{e}_s+\mathbf{g}}{1+\mathbf{g}^*\mathbf{e}_s} \,\,\,\,\,\,\,\, \vert \mathbf{g}\vert \leq 1 \\ \\
\frac{1+\mathbf{ge}_s^*}{\mathbf{e}_s^* + \mathbf{g}^*} \,\,\,\,\,\,\,\, \vert \mathbf{g}\vert > 1
\end{cases}
\end{equation}
%
where $\mathbf{g} = \pmb{\gamma}/(1-\kappa)$ is the reduced shear. For each simulated galaxy, we assign an intrinsic ellipticity by rotating the observed ellipticity for that galaxy by a random angle on the sky, while conserving its magnitude $\vert\mathbf{e}\vert$. To be consistent with the CFHTLenS analysis, we adopt the weak lensing limit ($\vert\pmb{\gamma}\vert\ll1,\kappa\ll1$),hence $\mathbf{g}\approx\pmb{\gamma}$ and $\mathbf{e}\approx \mathbf{e}_s+\pmb{\gamma}$. We also add the multiplicative shear corrections by replacing $\pmb{\gamma}$ with $(1+m)\pmb{\gamma}$. We note that the observed ellipticity for a particular galaxy already contains information from the lensing by large scale structure, but the random angle rotation makes this contribution second order in $\kappa$. Consistent with the weak lensing approximation, the lensing signal from the simulations is first order in $\kappa$ and hence the randomly rotated observed ellipticities can be effectively considered as intrinsic ellipticities. We analyze the simulations in the same way as we analyzed CFHTLenS data, constructing the simulated $\kappa$ maps as explained in \S\ref{cfhtdatareduction}. These final simulation products are then processed together with the $\kappa$ maps obtained from the data to calculate the confidence intervals on the parameter triplet $(\Omega_m,w,\sigma_8)$.

%AP: changed comoving distance symbol to chi; knocked out the ray tracing figure; clarified the signal double counting issue

%%%%%%%%%%%%%%%%%%%%%%%%%% METHODS %%%%%%%%%%%%%%%%%%%%%%%%%%%%%%%%%%%%%%%%%%%%%%%%%%%%%%

\section{Statistical methods}
The goal of this section is to describe the framework in which we combine the CFHT data and our simulations in order to derive the constraints on the cosmological parameter triplet $(\Omega_m,w,\sigma_8)$; we measure a set of statistical descriptors from the data and the simulations, which will then be compared in a Bayesian framework in order to compute parameter confidence intervals.

\subsection{Descriptors}
The statistical descriptors we consider in this work are the Minkowski Functionals (MFs) and the low order moments (LM) of the convergence field. The three MFs $(V_0,V_1,V_2)$ are topological descriptors of the convergence field $\kappa(\pmb{\theta})$, which probe respectively the area, perimeter and genus characteristic of the $\kappa$ excursion sets $\Sigma_{\kappa_0}$, defined as $\Sigma_{\kappa_0}=\{\kappa>\kappa_0\}$. Following \citep{Petri2013,MinkJan} we use the following local estimators to measure the MFs from the $\kappa$ maps. 
%
\begin{equation*}
\label{v0meas}
V_0(\kappa_0)=\frac{1}{A}\int_A\Theta(\kappa(\pmb{\theta})-\kappa_0)d\pmb{\theta},
\end{equation*}
\begin{equation}
\label{v1meas}
V_1(\kappa_0)=\frac{1}{4A}\int_A\delta(\kappa(\pmb{\theta})-\kappa_0)\sqrt{\kappa_x^2+\kappa_y^2}d\pmb{\theta},
\end{equation}
\begin{equation*}
\label{v2meas}
V_2(\kappa_0)=\frac{1}{2\pi A}\int_A\delta(\kappa(\pmb{\theta})-\kappa_0)\frac{2\kappa_x\kappa_y\kappa_{xy}-\kappa_x^2\kappa_{yy}-\kappa_y^2\kappa_{xx}}{\kappa_x^2+\kappa_y^2}d\pmb{\theta}.
\end{equation*}
%
Where $A$ is the total area of the fields of view and the notation $\kappa_{x,y}$ indicates gradients of the $\kappa$ field, which we evaluate using finite differences. The first Minkowski functional, $V_0$, is equivalent to the cumulative one--point PDF of the $\kappa$ field, $\partial V_0$ (which can be obtained by differentiation $\partial V_0(\kappa_0)=dV_0(\kappa_0)/d\kappa_0$), while $V_1,V_2$ are sensitive to the correlations between nearby pixels. In addition to these topological descriptors, we consider a set of low order moments (LM) of the convergence field (two quadratic, three cubic and four quartic), which are defined in the following way
%
\begin{equation}
\label{momentestimator}
\begin{matrix}
\mathrm{LM_2}: \sigma_{0,1}^2 = \langle\kappa^2\rangle,\langle\vert\nabla\kappa\vert^2\rangle, \\ \\
\mathrm{LM_3}: S_{0,1,2} = \langle\kappa^3\rangle,\langle\kappa\vert\nabla\kappa\vert^2\rangle,\langle\kappa^2\nabla^2\kappa\rangle, \\ \\
\mathrm{LM_4}: K_{0,1,2,3} = \langle\kappa^4\rangle,\langle\kappa^2\vert\nabla\kappa\vert^2\rangle,\langle\kappa^3\nabla^2\kappa\rangle,\langle\vert\nabla\kappa\vert^4\rangle.
\end{matrix}
\end{equation}
%

If the $\kappa$ field were Gaussian, one could express all the Minkowski Functionals in terms of the LM$_2$ moments, as expected from the fact that, for a Gaussian field, the only meaningful statistics are the quadratic ones. In reality, weak lensing convergence fields are non--Gaussian and the MF and LM descriptors are not guaranteed to be equivalent. \citep{Munshi12,Matsubara10} studied a perturbative expansion of the MF descriptors in powers of the $\kappa$ field standard deviation $\sigma_0$, which, when truncated at order $O(\sigma_0^2)$, can be expressed completely in terms of the LM up to quartic order. Such perturbative series, however, have been shown not to converge \citep{Petri2013} unless the weak lensing fields are smoothed with windows of size $\geq 15^\prime$. Because of this, throughout this work, we treat MF and LM as separate statistical descriptors. The LM we consider are restricted to the ones that appear in the perturbative expansion of the Minkowski functionals. The LM$_4$ moments which include derivatives of the $\kappa$ field ($K_{1,2,3}$) each emphasize a different set of trispectrum quadrilateral shapes. For example, $K_2$ emphasizes shapes where one side of the quadrilateral is large and $K_4$ emphasizes shapes close to rectangular. There are other moments which include derivatives besides the ones we have considered. In the future we will investigate whether there is additional constraining power in quartic moments not considered here.

In addition to the MF and LM descriptors, we consider the $\kappa$ angular power spectrum $P_l$ defined as
\begin{equation}
\label{powerspectrum}
\langle\tilde{\kappa}(\mathbf{l})\tilde{\kappa}(\mathbf{l}')\rangle=(2\pi)^2\delta_D(\mathbf{l}+\mathbf{l}')P_l
\end{equation}  
%
%AP: independent --> separate
%
where $\tilde{\kappa}(\mathbf{l})$ is the Fourier transform of the $\kappa$ field and $\delta_D$ is the usual Dirac delta function. While the use of this statistic is not a novelty, we want to compare the results we get using MF and LM to ones already present in the literature which are based on the power spectrum. A summary of the statistical descriptors used is presented in Table \ref{desctable}. 
%
\begin{table}
\begin{tabular}{c|c|c} \hline
Descriptor & Details & $N_b$ \\ \hline
$V_0,V_1,V_2$ (MF) & $\kappa_0\in[-0.04,0.12]$ & 50 \\
Power Spectrum (PS) & $l \in [300,5000]$ & 50 \\
Moments (LM) & -- & 9 \\
\end{tabular}
\caption{Summary of the descriptors we used, along with the specifications and the number of bins $N_b$ used}
\label{desctable}
\end{table}
%


When measuring statistical features on $\kappa$ maps, particular attention must be given to the effect of masked pixels. This does not pose a particular problem for the MFs and LM statistics, since the estimators in (\ref{v1meas}) and (\ref{momentestimator}) are well defined over all the non--masked region, with the exception of the few pixels that are close to the mask boundaries. Masking effects on Minkowski Functionals constraints with CFHTLenS have been studies extensively by \citep{CFHTMasato}. The situation is more complicated for the Power Spectrum measurements, that require the evaluations of Fourier transforms and hence rely on the value of every pixel in the map. Although sophisticated interpolation approaches over the masked regions have been studied (see for example \citep{VplasInterpolation}), for the sake of simplicity we put a value of 0 in each masked pixel, arguing that, given the typical angular size of the masked regions we consider, this will have little effect on the Power Spectrum at the range of multipoles in Table \ref{desctable}. We believe that the way we deal with masked sky regions is very robust, since we apply the same masks to our simulations. 

%AP: Stressed the fact that we apply the same mask to the simulations; added reference to Yoshida/Shirasaki 

\subsection{Cosmological parameter inferences}
\label{cosmostats}

In this section we give a brief outline of the statistical framework adopted for computing the cosmological parameter confidence levels from the CFHTLenS observations. We make use of the MF and LM statistical descriptors outlined in the previous section. We refer to $M_i^r(\mathbf{p})$ as the measured descriptor from a realization $r$ of one of our simulations with a choice of cosmological parameters $\mathbf{p}$, and to $D_i$ as the measured descriptor from the CFHTLenS data. In this notation, $i$ is an index that refers to the particular bin on which the descriptor is evaluated (for example $i$ can range from 0 to 9 for the LM statistic and from 0 to $N_b-1$ for a Minkowski Functional measured on $N_b$ different excursion sets). Once we make an assumption for the data likelihood $\mathcal{L}_d(D_i\vert \mathbf{p})$ and for the parameter priors $\Pi(\mathbf{p})$, we can use Bayes theorem to compute the parameter likelihood $\mathcal{L}_p$ as follows

\begin{equation}
\label{parameterlikelihood}
\mathcal{L}_p(\mathbf{p}\vert D_i) = \frac{\mathcal{L}_d(D_i\vert \mathbf{p})\Pi(\mathbf{p})}{N_{\mathcal{L}}}
\end{equation}
%
where $N_{\mathcal{L}}$ is a $\mathbf{p}$--independent constant that ensures the proper normalization for $\mathcal{L}_p$; we make the usual assumption that the data likelihood $\mathcal{L}_d(D_i\vert \mathbf{p})$ is Gaussian

%AP: what kind of quantitative test should I implement? We can use Anderson-Darling to check how gaussian the pdf in each bin is, but this doesn't tell anything on the bin correlations. Also which bins shall I choose? There is a problem if the likelihood is non-Gaussian: we don't know what to do. Basically this is the only functional form with which we can model the PDF and bin correlations analytically, and compute the parameter likelihood semi-analytically, otherwise we need to dig into more sophisticated non--parametric methods in which I am not that proficient yet. 

\begin{equation}
\label{datalikelihood}
\begin{matrix}
\mathcal{L}_d(D_i\vert \mathbf{p}) = ((2\pi)^{N_b}\det{\mathbf{C}})^{-1/2} e^{-\frac{1}{2}\chi^2(D_i\vert \mathbf{p})} \\ \\
\chi^2(D_i\vert \mathbf{p}) = \mathbf{(D - M(p))C^{-1}(D-M(p))}.
\end{matrix}
\end{equation} 
%

The simulated descriptors $\mathbf{M(p)}$ are measured from an average over realizations
\begin{equation}
M_i(\mathbf{p}) = \frac{1}{R}\sum_{r=1}^R M_i^r .
\end{equation}
%
The covariance matrix 
\begin{equation}
\label{datacov}
C_{ij} = \frac{1}{R-1} \sum_{r=1}^R [M_i^r(\mathbf{p}_0)-M_i(\mathbf{p}_0)][M_j^r(\mathbf{p}_0)-M_j(\mathbf{p}_0)]
\end{equation}
%
is measured from a simulation set based on 50 independent $N$--body simulations with parameters $\mathbf{p}_0=(0.26,-1.0,0.8)$ and is assumed to be model-independent. Thus the normalization constant in (\ref{datalikelihood}) is also model--independent. When computing parameter constraints from CFHTLenS weak lensing data alone, we make a flat prior assumption for $\Pi(\mathbf{p})$. Parameter inferences are made estimating the location of the maximum of the parameter likelihood in (\ref{parameterlikelihood}), which we call $\mathbf{p}_{ML}(D_i)$ as well as its confidence contours. A $N\sigma$--confidence contour of $\mathcal{L}_p(\mathbf{p}\vert D_i)$ is defined to be the subset of points in parameter space on which the likelihood has a constant value $c_N$ and 
\begin{equation}
\label{ennesigma}
\int_{\mathcal{L}>c_N} \mathcal{L}_p(\mathbf{p}\vert D_i) d\mathbf{p} = \frac{1}{\sqrt{2\pi}}\int_{-N}^N dx e^{-x^2/2}.
\end{equation}
%

Given the low dimensionality of the parameter space we consider $(N_p=3)$ we are able to evaluate the parameter likelihood (\ref{parameterlikelihood}) on a finely spaced $100\times100\times100$ mesh within the prior window $\Pi(\mathbf{p})$, and hence we are able to evaluate the likelihood maximum $\mathbf{p}_{ML}(D_i)$ and the contour levels $c_N$ directly without the need to use more sophisticated MCMC methods. We know how to evaluate the data likelihood (\ref{datalikelihood}) on the simulation grid $\mathbf{p}_s$, but more work needs to be done to interpolate $M_(\mathbf{p})$ to an arbitrary intermediate point. We use a Radial Basis Function (RBF) interpolation scheme. We approximate the model descriptor as
\begin{equation}
M(\mathbf{p}) = \sum_{s=1}^N \lambda_s\phi(\vert\mathbf{p}-\mathbf{p}_s\vert)
\end{equation}
%
where $\phi$ has been chosen as a multiquadric function $\phi(r)=\sqrt{1+(r/r_0)^2}$ with $r_0$ chosen as the mean Euclidean distance between the points in the simulated grid $\mathbf{p}_s$. The constant coefficients $\lambda_s$ can be determined imposing the $N$ constraints $M(\mathbf{p}=\mathbf{p}_s)=M(\mathbf{p_s})$, which enforce the fact that the interpolation should be exact at the simulated points. The interpolation computations are conveniently performed using the Scipy library \citep{scipy}. 

We studied the accuracy of emulator, built with the CFHTemu1 simulations, by interpolating the convergence features to the fiducial parameter setting $(\Omega_m,w,\sigma_8)=(0.26,-1.0,0.8)$ and comparing the result to the one expected from the CFHTcov simulations. Figure \ref{emulatorAccuracy} shows that our Power Spectrum emulator has accuracy better than 10\% for the lower multipoles and comparable to 1\% for the higher multipoles. The Minkowski Functionals emulator has accuracy $\lesssim$10\% for the first 30 bins (which correspond to $\kappa$ values in $[-0.04,0.08]$) and deteriorates due to numerical noise for the remaining 20 bins; we take care of these inaccuracies in our dimensionality reduction framework, which we explain in the next section. 

\begin{figure}
\begin{center}
\includegraphics[scale=0.45]{Figures/emulator_accuracy.eps}
\end{center}
\caption{Accuracy of the emulator based on the CFHTemu1 simulations: we show the difference between the interpolated feature at the fiducial parameter setting and the expected feature from the CFHTcov simulations, in units of the standard deviation in each bin $i$ (determined from the diagonal elements of the CFHTcov covariance matrix). We show the accuracy results for the Power Spectrum (red) and Minkowski Functionals $V_0$ (green), $V_1$ (blue) and $V_2$ (black).}
\label{emulatorAccuracy}
\end{figure}  

\subsection{Dimensionality reduction}
\label{pcasection}
The main goal of this work is constraining the cosmological parameter triplet $(\Omega_m,w,\sigma_8)$ using the CFHTLenS data. When the question on the accuracy of the $N\sigma$ contours is considered, particular attention must be paid to the effect of binning choices on contour sizes computed with equations (\ref{parameterlikelihood})--(\ref{ennesigma}). It is known that the choice of the number of bins, $N_b$, can have a non-negligible effect on the contour sizes (see \citep{Petri2013} for an example with simulated datasets); in order for our results to be robust under this effect, we adopt a Principal Component Analysis (PCA) approach. Our main physical motivation for this approach is that, albeit we need to specify $N_b$ numbers in order to fully characterize a binned descriptor, we suspect that the majority of the constraining information (of a particular feature) is contained in a limited number of linear combinations of its binning.  In the framework adopted by \citep{coyote2}, for example, the authors find that the majority of the cosmological information in the matter Power Spectrum is contained in only 5 selected linear combinations of the multipoles. Because of this, we believe that dimensionality reduction techniques such as PCA can help in delivering constraints that are independent on the number of bins $N_b$ originally chosen. 

In order to compute the Principal Components of our feature space, we consider the CFHTemu1 simulations, which sample the cosmological parameter space at the $N=91$ points outlined in Table \ref{designtable} and allow us to compute the $N\times N_b$ model matrix $M_{pi}=M_i(\mathbf{p})$. Following a standard procedure (see \citep{astroMLText} for example), we then derive the whitened model matrix $\tilde{M}_{pi}$, defined by subtracting the mean in each bin, and normalizing it by its variance. Next we proceed with a singular value decomposition (SVD) of $\mathbf{\tilde{M}}$
\begin{equation}
\label{svd}
\mathbf{U}\mathbf{S} \mathbf{V}^T=\frac{\mathbf{\tilde{M}}}{\sqrt{N-1}}
\end{equation}   
%
where $S_{ij}=S_i\delta_{ij}$ is a diagonal matrix and $V^T_{ij}$ is the $i$--th principal component of $\mathbf{\tilde{M}}$, with the index $j$ ranging from $0$ to $N_b-1$. 

To rank the Principal Components $V^T$ in order of importance, we note that the diagonal matrix $\mathbf{S}^2$ is nothing more than the diagonalization of the model covariance (not to be confused with the feature covariance in (\ref{datacov}))
\begin{equation}
\frac{1}{N-1}\mathbf{\tilde{M}}^T\mathbf{\tilde{M}} = \mathbf{V}\mathbf{S}^2\mathbf{V}^T.
\end{equation} 
%
We follow the standard interpretation of PCA components, stating that the only meaningful components $V^T_i$ in the analysis (i.e. the ones that contain the relevant cosmological information) are those corresponding to the largest eigenvalues $S^2_{i}$, with the smallest eigenvalues corresponding to noise in the model, due to numerical inaccuracies in the simulation pipeline. We expect our constraints to be stable with respect to the number of components, once enough components have been included. Using the fact that different Principal Components are orthogonal, we perform a PCA projection on our feature space by whitening the features and computing the dot product with the principal components, keeping only the first $n$ components
\begin{equation}
\label{pcaprojection}
M(n)_{i}^r = V^T(n)_{ij}\tilde{M}_j^r \,\,\,\, ; \,\,\,\,  D(n)_i = V^T(n)_{ij}\tilde{D}_j
\end{equation}
%
where we indicate with $V^T(n)$ the truncation of $V^T$ to the first $n$ rows. The dimensionality reduction comes from the fact that we expect most of the cosmological information to be contained in a number of components $n<N_{b}$. 

Looking at the dimensionality issue from a more geometric perspective, we expect our feature space to be a 3-dimensional manifold embedded in a $N_b$--dimensional space. The dimensionality reduction problem is equivalent to the accurate reconstruction of the coordinate chart of this feature manifold. As outlined in \citep{astroMLText}, the coordinate chart constructed with the PCA projection in (\ref{pcaprojection}) is accurate for reasonably flat feature manifolds. When curvature effects become important, more advanced projection techniques (such as Locally Linear Embedding) have to be employed. We have flexibility on the flatness assumption because we allow the number of principal components to be greater than 3. 

%AP: Addressed dimensionality issues 

%%%%%%%%%%%%%%%%%%%%%%%%%% RESULTS %%%%%%%%%%%%%%%%%%%%%%%%%%%%%%%%%%%%%%%%%%%%%%%%%%%%%%

%
\begin{figure*}
\includegraphics[scale=0.4]{Figures/pca_components.eps}
\caption{Principal Components of the Power Spectrum(red), $V_0$ (blue), $V_1$ (green), $V_2$ (black) and the Moments (orange) feature spaces; the left plot shows the magnitudes of the PCA eigenvalues $S_i^2$, the right plot shows their cumulative sum. A dashed black line has been drawn in correspondence of $n=3$ components}
\label{pcafig}
\end{figure*}
%
\begin{figure*}
\includegraphics[scale=0.4]{Figures/robustness_pca_Omega_m-sigma8.eps}
\caption{PCA projection dependence of the $1\sigma$ contours in the $(\Omega_m,\sigma_8)$ plane obtained from a mock observation constructed with the CFHTcov simulations; the different panels refer to the descriptors (from left to right, top to bottom) $V_0$, $\partial V_0$(PDF), $V_1$, $V_2$, Power Spectrum and Moments}
\label{robustnessfig}
\end{figure*}
%

\section{Results}
\label{results}
This section describes the results we have obtained and is organized as follows. We first use our simulations to perform a robustness analysis of the parameter confidence intervals with respect to the number of PCA components used in the projection. We then show the cosmological constraints from the CFHTLenS data. Because of the relatively small size of this survey, degeneracy in the parameters can have undesired effect on the constraints. To mitigate the effect of degeneracies, in addition to the usual $(\Omega_m,w,\sigma_8)$ parametrization, we consider a slightly different one, built with the parameter triplet $(\Omega_m,w,\Sigma_8)$ with $\Sigma_8(\alpha)=\sigma_8(\Omega_m/0.27)^\alpha$. While $\Omega_m$ and $\sigma_8$ are poorly constrained by CFHTLenS due to degeneracies, the $\Sigma_8(\alpha)$ combination lies on the smallest variance direction of $\mathcal{L}(\Omega_m,\sigma_8)$ for a suitable choice of $\alpha$, and hence has a much smaller relative uncertainty. We can derive this optimal value of $\alpha$ from the full three dimensional likelihood $\mathcal{L}(\Omega_m,w,\sigma_8)$, from which we can compute the expectation values
\begin{equation}
\mathds{E}(\alpha) = \langle\Sigma_8(\alpha)\rangle \,\,\, ; \,\,\, \mathds{V}(\alpha) = \langle(\Sigma_8(\alpha)-\mathds{E}(\alpha))^2\rangle
\end{equation}
%
and minimizing the ratio $\sqrt{\mathds{V}}/\mathds{E}$ with respect to $\alpha$. This procedure yields a value $\alpha\approx0.55$ for the statistical descriptors that we consider, consistent  with what is found in the literature. 

We show the constraints in both the $(\Omega_m,\sigma_8)$ and $(w,\Sigma_8)$ planes. We also show the $\Sigma_8$ marginalized likelihood. We finally study the effect of combining different descriptors, tightening the constraints. A summary with the complete set of results, along with the relevant Ffgures, is shown in Table \ref{summarytable}. 

%AP: Justified value of alpha=0.55

\subsection{Robustness}
%
Here we give evidence that the cosmological constraints derived in this paper are numerically robust, i.e. they are reasonably stable once we consider a large enough number $n$ of Principal Components. Figure \ref{pcafig} shows the PCA eigenvalues that result from the SVD decomposition of our feature spaces, as well as the cumulative sum of these eigenvalues, normalized to 1. Figure \ref{robustnessfig} shows the dependence of the $(\Omega_m,\sigma_8)$ constraints on the number of principal components $n$, and clearly indicates that we only need a limited number of components in order to capture the cosmological information contained in our descriptors. The number of components with significant information content depends on the particular descriptor considered. This also addresses the inaccuracy of the MF emulator at high thresholds pointed out in Figure \ref{emulatorAccuracy}. By keeping a limited number of principal components we are able to prevent the inaccurate high--threshold bins, that have a low constraining power, from contributing to the parameter confidence levels.    

%AP: Added inline description of the figures

\subsection{Cosmological constraints}
%
We make use of equations (\ref{parameterlikelihood})-(\ref{ennesigma}) to compute the $1\sigma$ constraints on different combinations of cosmological parameters: in Figures \ref{contours3single}, \ref{contoursMoments} we show the constraints in the $(\Omega_m,\sigma_8)$ plane, in Figure \ref{contours3singleRep} we focus on the $(w,\Sigma_8)$ plane. Additionally we show the $\Sigma_8$ likelihood function (marginalized over $\Omega_m$ and $w$) in Figure \ref{likelihoodSi8single}. We discuss these results in \S~\ref{discussion}. 

\begin{figure*}
\begin{center}
\includegraphics[scale=0.45]{Figures/contoursOmega_m-sigma8_single.eps}
\includegraphics[scale=0.45]{Figures/contoursmockOmega_m-sigma8_single.eps}
\end{center}
\caption{$1\sigma$ constraints on the $(\Omega_m,\sigma_8)$ parameter doublet using the Power Spectrum (red), $V_0$ (blue), $V_1$ (green), $V_2$ (black) and Moments (orange) statistics; we show the constraints from the data (left panel) and from a mock observation constructed using the mean of 1000 realizations in CFHTcov simulations as data (right panel). The contours are calculated from the parameter likelihood function $\mathcal{L}$ marginalized over $w$. The parentheses near the descriptor label refer to the number of principal components.}
\label{contours3single}
\end{figure*}

\begin{figure*}
\begin{center}
\includegraphics[scale=0.45]{Figures/contours_momentsOmega_m-sigma8.eps}
\includegraphics[scale=0.45]{Figures/contours_momentsOmega_m-sigma8smooth.eps}
\end{center}
\caption{$1\sigma$ constraints on the $(\Omega_m,\sigma_8)$ parameter doublet using the Moments, with different colors corresponding to different moment combinations: we show the results obtained using the one--point moments $\sigma_0^2,S_0,K_0$ (black, left and right panels). In the left panel we show the constraints obtained adding moments of gradients to the one--point moments, while in the right panel we combine the one--point momente measured at different smoothing scales.}
\label{contoursMoments}
\end{figure*}

\begin{figure}
\begin{center}
\includegraphics[scale=0.45]{Figures/contoursw-Sigma8Om055_single.eps}
\end{center}
\caption{$1\sigma$ constraints on the $(w,\Sigma_8)$ parameter doublet from the CFHTLenS data, obtained with the Power Spectrum (red), $V_0$ (blue), $V_1$ (green), $V_2$ (black) and Moments (orange) statistics; we show the constraints from the data. The contours are calculated from the parameter likelihood function $\mathcal{L}$ marginalized over $\Omega_m$. The parentheses near the descriptor label refer to the number of principal components.}
\label{contours3singleRep}
\end{figure}

\begin{figure}
\begin{center}
\includegraphics[scale=0.45]{Figures/contoursSigma8Om055_single.eps}
\end{center}
\caption{$\Sigma_8$ parameter likelihood calculated from the CFHTLenS data using the Power Spectrum (red), $V_0$ (blue), $V_1$ (green), $V_2$ (black) and Moments (orange) statistics; the likelihood has been marginalized over $\Omega_m$ and $w$. The parentheses near the descriptor label refer to the number of principal components.}
\label{likelihoodSi8single}
\end{figure}

\begin{table*}
\begin{tabular}{c|c|c||c}
Parameters & Descriptors & Short description & Relevant Figures \\ \hline \hline
$(\Omega_m,\sigma_8)$ & PS,$V_0,V_1,V_2$,LM & Stability of contours & \ref{robustnessfig} \\ \hline 
$(\Omega_m,\sigma_8)$ & PS(3),$V_0(5),V_1(20),V_2(20)$,LM(9) &\pbox{20cm}{$1\sigma$ constraints from CFHTLenS \\ and mock observations}  & \ref{contours3single},\ref{contours3single}b \\ \hline
$(\Omega_m,\sigma_8)$ & $(\sigma_i^2,S_i,K_i)$ & \pbox{20cm}{$1\sigma$ constraints from CFHTLenS \\ using $\kappa$ Moments \\ combined at different $\theta_G$}  & \ref{contoursMoments},\ref{contoursMoments}b \\ \hline
$(w,\Sigma_8)$ & PS(3),$V_0(10),V_1(10),V_2(10)$,LM(9) & contours from CFHTLenS & \ref{contours3singleRep} \\ \hline 
$\Sigma_8$ & PS(3),$V_0(10),V_1(10),V_2(10)$,LM(9) & $\mathcal{L}(\Sigma_8)$ from CFHTLenS & \ref{likelihoodSi8single} \\ \hline
$(\Omega_m,\sigma_8)$ & PS(3)$\times V_0(5)\times V_1(20)\times V_2(20)\times$LM(9) & \pbox{20cm}{constraints from CFHTLenS \\ combining statistics} & \ref{contours3combined} \\ \hline
$(w,\Sigma_8)$ & PS(3)$\times V_0(10)\times V_1(10)\times V_2(10)\times$LM(9) & \pbox{20cm}{constraints from CFHTLenS \\ combining statistics} & \ref{contours3combined}b \\ \hline 
$\Sigma_8$ & PS(3)$\times V_0(10)\times V_1(10)\times V_2(10)\times$LM(9) & \pbox{20cm}{$\mathcal{L}(\Sigma_8)$ from CFHTLenS \\ combining statistics} & \ref{likelihoodSi8cross} \\ \hline
\end{tabular}
\caption{Summary table of our results}
\label{summarytable}
\end{table*}
%

\begin{figure*}
\begin{center}
\includegraphics[scale=0.45]{Figures/contoursOmega_m-sigma8_cross.eps}
\includegraphics[scale=0.45]{Figures/contoursw-Sigma8Om055_cross.eps}
\end{center}
\caption{Combined $1\sigma$ constraints on the $(\Omega_m,\sigma_8)$ (left panel) and $(w,\Sigma_8)$ (right panel) doublets, using the PS (red), PS$\times$Moments (green), MFs (blue), MFs$\times$Power Spectrum (black) and MFs$\times$Power Spectrum$\times$Moments (orange). The likelihood function has been marginalized over $w$ (left panel) and $\Omega_m$ (right panel). The parentheses near the descriptor labels refer to the number of principal components.}
\label{contours3combined}
\end{figure*}

\begin{figure}
\begin{center}
\includegraphics[scale=0.45]{Figures/contoursSigma8Om055_cross.eps}
\end{center}
\caption{$\Sigma_8$ parameter likelihood calculated from the CFHTLenS data using the PS (red), PS$\times$Moments (green), MFs (blue), MFs$\times$Power Spectrum (black) and MFs$\times$Power Spectrum$\times$Moments (orange); the likelihood function has been marginalized over $\Omega_m$ and $w$. The parentheses near the descriptor label refer to the number of principal components.}
\label{likelihoodSi8cross}
\end{figure}

\subsection{Combining statistics}
Combinating of different statistics can tighten the constraints on cosmology. Previous work in this direction includes \citep{Companion}, where the authors combine the power spectrum and peak count statistics and  \citep{CFHTFu} where the authors combine quadratic statistics with the skewness of the CFHTLenS $\kappa$ fields. 

The procedure we adopt is the following:  Consider two descriptors, $d_{1,i},d_{2,i}$ where the index $i$ corresponds to the bin number of the particular descriptor. We compute the single descriptor constraints as in Figure \ref{robustnessfig} and we decide which is the minimum number of principal components $n_{1,2}$ needed for the constraints to be stable. We then construct the vector $d_{1\times2} = (d_1(n_1),d_2(n_2))$ and consider this as the combined feature vector. This procedure allows us to take the cross covariance between different descriptors. We show in Figure \ref{contours3combined} the combined CFHTLenS constraints in the $(\Omega_m,\sigma_8)$ and $(w,\Sigma_8)$ planes, and in Figure \ref{likelihoodSi8cross} the combined $\Sigma_8$ likelihood. We discuss our findings in the next section. 

%AP: Added the statistics combinations Zoltan suggested

%%%%%%%%%%%%%%%%%%%%%%%%%% DISCUSSION %%%%%%%%%%%%%%%%%%%%%%%%%%%%%%%%%%%%%%%%%%%%%%%%%%%%%%

\section{Discussion}
\label{discussion}

In this section we discuss the results outlined in section \S~\ref{results}, with particular focus on the constraints on cosmology. 

As pointed out in \S~\ref{pcasection}, the choice of the number of bins, $N_b$, is an important issue. In order for our results to be insensitive to $N_b$, we adopt a PCA projection technique to reduce the dimensionality of our feature spaces. In the left panel of Figure \ref{pcafig} we see that the PCA eigenvalues for our descriptors decrease about 4 orders of magnitude from $n=1$ to $n=3$, and looking at the right panel we see that we capture more than 99\%  of the feature variance even if we cut the PCA projection to the first 3 components. 

This does not mean that all the cosmological information is captured by the first 3 PCA components, and this is why we make a stability check by plotting the $1\sigma$ contour sizes as a function of $n$ in Figure \ref{robustnessfig}. Looking at the plot we can observe that, while the first 3 components capture essentially all the information contained in the Power Spectrum, this is not true for the remaining set of statistics. In particular we find that at least 5 components are necessary for $V_0$, and at least 20 components for $V_1,V_2$, in order for the $(\Omega_m,\sigma_8)$ contours to be stable at a $\sim$5\% level. All the 9 components are necessary for the Moments contours to be stable. 

Matters are slightly different when we study the $(w,\Sigma_8)$ constraints: in this case we find that the optimal choice for the three MFs, $(V_0,V_1,V_2)$, is $n=10$, while the Power Spectrum and Moments components remain at, respectively, 3 and 9 (figures supporting this choice not shown).

We now discuss the main scientific findings of this work. In Figure \ref{contours3single} we show the $1\sigma$ constraints on the $(\Omega_m,\sigma_8)$ doublet from the CFHTLenS data. The MF constraints are heavily biased towards the low--$\sigma_8$, high--$\Omega_m$ region. The best current values of $(\Omega_m,\sigma_8)$  \citep{PlanckXVI2013} lie beyond the 99\% likelihood contours (not shown). This may be due to uncorrected systematics in the CFHTLenS data, amplified by the $(\Omega_m,\sigma_8)$ degeneracy. When we try to constrain mock observations based on simulations (shown in the left panel of Figure \ref{contours3single}), we recover the correct position of the $1\sigma$ contours.  It is important to note, however, that the mock observations to which the right panel of Figure \ref{contours3single} refers, were built with the mean of $R=1000$ realizations of the CFHTcov simulations. We found that it is possible to find some realizations for which the best fit for $(\Omega_m,\sigma_8)$ lies in the lower right corner, though the likelihood of this happening is very small ($\lesssim1\%$). 

%Added that  the reason we think MFs are biased is that the best current values of Om, s8 lie beyond the 99% likelihood contours for MFs. Changed "We believe that this is due to uncorrected systematics" to "This may be due to"  .  Added that bias is amplified by degeneracy.  Morgan 2/16/2015

We observe that the Moments give the tightest constraint on $(\Omega_m,\sigma_8)$.  This constraint is also unbiased in the sense that it includes the best current values for these parameters within 1$\sigma$. The fact that the Moments are useful for deriving unbiased cosmological constraints has been noted in previous work  \citep{PetriSpurious}. 

In order to determine the origin of the tight bounds derived from moments, we studied the individual contributions of each of the moments to the constraining power. Figure \ref{contoursMoments} shows the evolution of the $(\Omega_m,\sigma_8)$ constraints as we add higher order moments to the feature set. Since we are constraining  3 cosmological parameters, we start by considering the set of the three one--point moments $(\sigma_0^2,S_0,K_0)$ and then add the remaining six moments of gradients one by one, starting from the quadratic moments. We see that the biggest improvement on the parameter bounds comes from including quartic moments of gradients (i.e. $K_i$ with $i\ge1$) in the feature set. This might explain why \citep{CFHTFu} find only unimpressive ($\sim10\%$) contour tightening when adding the skewness to quadratic statistics, since the main improvement comes from higher moments of $\kappa$ derivatives. Our result agrees with  \citep{moments4}  which found that the kurtosis of the shear field can help in breaking degeneracies between $\Omega_m$ and $\sigma_8$. We also observe that the improvement obtained by combining the one--point moments at different smoothing scales $\theta_G$, is not as large as that obtained by considering the moments of gradients.      

The bias in the $(\Omega_m,\sigma_8)$ constraints is amplified by the cosmological degeneracy of these parameters. To mitigate this effect, we consider the combination of $\Omega_m$ and $\sigma_8$ that lies along the non-degenerate direction, namely $\Sigma_8=\sigma_8(\Omega_m/0.27)^{0.55}$. Figure \ref{likelihoodSi8single} shows the $1\sigma$ constraints for the $(w,\Sigma_8)$ doublet, as well as the marginalized $\Sigma_8$ likelihood from the CFHTLenS data. The CFHTLenS survey is not sufficient to constrain $w$, but it does well in determining the $\Sigma_8$ combination to a value of $\Sigma_8=0.75\pm0.04(1\sigma)$ using the full descriptor set, in agreement with values previously published by the CFHTLenS collaboration \citep{CFHTKilbinger}. 

%"using either of the MFs" changed to using the full descriptor set Morgan 2/16/2015

Regarding the parameter biases, our results are in accordance with  \citep{PetriSpurious}, namely, that unaccounted systematics result in larger parameter biases when the constraints are derived from the MFs, and that the LM statistic is less biased. However, for the CFHTLenS data the MFs can effectively constrain the non-degenerate direction in parameter space, $\Sigma_8$, where the amplifying effects of degeneracy are mitigated (Figure \ref{likelihoodSi8single}). 


%MF constraints on Om, s8 are biased. However MFs can effectively constrain the non-degenerate direction, S8 for CFHTLenS

We next studied whether the combination of different statistical descriptors can help in tightening the cosmological constraints; we show the effects of some of these combinations in Figures \ref{contours3combined} and \ref{likelihoodSi8cross}. The left panel of Figure \ref{contours3combined} shows that, although combining the Power Spectrum with the Minkowski Functionals and Moments helps in shrinking the $(\Omega_m,\sigma_8)$ constraints, it does not help in reducing the inherent parameter bias. In the right panel of Figure \ref{contours3combined} we can see that even with these statistics combined, $w$ remains essentially unconstrained. Figure \ref{likelihoodSi8cross} shows that the $\Sigma_8$ combination is already well constrained by any of the descriptors alone, without the need of combining different descriptors.   

%\newpage 
%%%%%%%%%%%%%%%%%%%%%%%%%% CONCLUSIONS %%%%%%%%%%%%%%%%%%%%%%%%%%%%%%%%%%%%%%%%%%%%%%%%%%%%%%

\section{Conclusions}

In this final section we summarize the main conclusions of this work:
\begin{itemize}
\item We find that the Moments of the $\kappa$ field provide the tightest constraint on the $(\Omega_m,\sigma_8)$ doublet from the CFHTLenS survey data.  Evidence of the unbiased nature of constraints from the Moments has been found in \citep{PetriSpurious}. We find that the biggest improvement in bounds on these parameter is achieved when we include the quartic moments of gradients in the feature set. This improvement cannot be obtained by combining one-point moments at different smoothing scales 
\item Although weak lensing surveys are a promising technique to constrain the DE equation of state parameter $w$, reasonable constraints cannot be obtained with the CFHTLenS survey alone, even when using additional sets of descriptors that go beyond the standard quadratic statistics.
\item When studying the cosmological information contained in the CFHTLenS data, special attention must be paid to the effect of residual systematic biases. While it seems these residual systematics are not important when constraining cosmology with the Power Spectrum alone, we find that these systematics need to be corrected to obtain sensible constraints on the $(\Omega_m,\sigma_8)$ doublet using the Minkowski Functionals. 
\item For the CFHTLenS data set, Minkowski functionals can effectively constrain the non-degenerate direction in parameter space, $\Sigma_8$, where the amplifying effects of degeneracy are mitigated.The Minkowski Functionals alone are sufficient to constrain the $\Sigma_8$ combination to a value of $\Sigma_8=0.75\pm0.04$ at $1\sigma$ significance level; this agrees with the value previously published by the CFHTLenS collaboration within $1\sigma$. Some tensions with Planck \citep{PlanckXVI2013} still remain.  
\end{itemize}

Possible future extensions of this work include simulating higher dimensional parameter spaces (possibly including varying Hubble parameter $H_0$ and the running of the DE equation of state $w_1$), and combining constraints from different cosmological probes (data from Planck for example), that can help in breaking the $\Omega_m,\sigma_8$ degeneracy, and application of our analysis to larger survey data sets. 

%%%%%%%%%%%%%%%%%%%%%%%%%% ACKNOWLEDGMENTS %%%%%%%%%%%%%%%%%%%%%%%%%%%%%%%%%%%%%%%%%%%%%%%%%%%%%%
 

\section*{Acknowledgements}
The simulations referenced in this work were performed at the NSF Extreme Science and Engineering Discovery Environment (XSEDE), supported by grant number ACI-1053575, and at the New York Center for Computational Sciences, a cooperative effort between Brookhaven National Laboratory and Stony Brook University, supported in part by the State of New York. This work was supported in part by the U.S. Department of Energy under Contract No. DE-AC02-98CH10886 and Contract No. DE-SC0012704.

\bibliography{ref}
\label{lastpage}
\end{document}
