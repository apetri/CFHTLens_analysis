%\documentclass[useAMS, usenatbib,usegraphicx,letter]{mn2e}
%\documentclass[11pt]{article}
\documentclass[reprint,aps,prd,superscriptaddress,showkeys]{revtex4-1}
\usepackage{epsfig,amsmath,natbib}

\usepackage{aas_macros}
\usepackage{amssymb}
 \usepackage{amsmath}
 \usepackage{dsfont}
 \usepackage{hyperref}
 \usepackage{color}
\hypersetup{
	colorlinks=false,
	citecolor=green
}
% \usepackage{graphicx}
% \usepackage{epstopdf}
% \usepackage{natbib}

\begin{document}

\title{CFHTLens Weak Lensing Emulator and Cosmological Constraints from the Minkowski Functionals and Moments}

\author{Andrea Petri}
\email{apetri@phys.columbia.edu}
\affiliation{Department of Physics, Columbia University, New York, NY 10027, USA}
\affiliation{Physics Department, Brookhaven National Laboratory, Upton, NY 11973, USA}

\author{Jia Liu}
\affiliation{Department of Astronomy, Columbia University, New York, NY 10027, USA}

\author{Zolt\'an Haiman}
\affiliation{Department of Astronomy, Columbia University, New York, NY 10027, USA}

\author{Lam Hui}
\affiliation{Department of Physics, Columbia University, New York, NY 10027, USA}

\author{Jan M. Kratochvil}
\affiliation{Astrophysics and Cosmology Research Unit, University of KwaZulu-Natal, Westville, Durban 4000, South Africa}

\author{Morgan May}
\affiliation{Physics Department, Brookhaven National Laboratory, Upton, NY 11973, USA}

\date{\today}

\label{firstpage}

\begin{abstract}
Weak lensing surveys have the potential... we consider CFHTLens \citep{CFHTFu}
\end{abstract}

\keywords{Weak Gravitational Lensing --- Data analysis --- Methods: analytical,numerical,statistical}

\maketitle


%%%%%%%%%%%%%%%%%%%%%%%%%% INTRO %%%%%%%%%%%%%%%%%%%%%%%%%%%%%%%%%%%%%%%%%%%%%%%%%%%%%%%%

\section{Introduction}
In this work...

%%%%%%%%%%%%%%%%%%%%%%%%%% DATA AND SIMULATIONS %%%%%%%%%%%%%%%%%%%%%%%%%%%%%%%%%%%%%%%%%%%%%%%%%%

\section{Data and simulations}

\subsection{CFHTLens data reduction}

\subsection{Simulation design}
In this paragraph we give a description of the method we used to sample the parameter space in our simulation effort. We wish to investigate the non--linear dependency of cosmological probes (in this work Minkowski Functionals and Moments of the $\kappa$ field) on the parameter triplet $\mathbf{p}=(\Omega_m,w,\sigma_8)$: in order to do this we sampled the $D$--dimensional ($D=3$ in this case) parameter space using an irregularly spaced grid, designed with a method similar to \citep{coyote2}. We limit the parameter sampling in a box of corners $\Omega_m\in[0.07,1],\,w\in[-3.0,0],\,\sigma_8\in[0.1,1.5]$ and we map this sampling box $\Pi$ into an hypercube of unit side; we want to contruct an irregularly spaced grid consisting of $N$ points $\mathbf{x}_i\in[0,1]^D$. Let a \textit{design} $\mathcal{D}$ be the set of this irregularly spaced $N$ points: we wish to find an optimal design, in which the points are spread as uniformly as possible inside the box. Following \citep{coyote2}, we choose our optimal design as the minimum of the cost function

\begin{equation}
\label{costfunction}
d(\mathcal{D}) = \frac{2D^{1/2}}{N(N-1)}\sum_{i<j}^N\frac{1}{\vert\mathbf{x}_i-\mathbf{x}_j\vert}
\end{equation} 
%
This problem is mathematically equivalent to the minimization of the Coulomb potential energy of $N$ unit charges in a unit box, which will make sure that the charges are as evenly spread as possible throughout the confining volume. Finding the optimal design $\mathcal{D}_m$ that minimized (\ref{costfunction}) can be computationally very demanding, and hence we decided to use a simplified approach that, although approximate, serves our purposes for the grid design. We use an iterative procedure:
\begin{enumerate}
\item We start from the diagonal design $\mathcal{D}_0$: $x_i^d\equiv i/(N-1)$
\item We shuffle the coordinates of the particles in each dimension independently $x_i^d = \mathcal{P}_d\left(\frac{1}{N-1},\frac{2}{N-1},...,1\right)$ where $\mathcal{P}_1,...,\mathcal{P}_D$ are random independent permutations of $(1,2,...,N)$
\item We pick a random particle pair $(i,j)$ and a random coordinate $d\in\{1,...,D\}$ and swap $x_i^d\leftrightarrow x_j^d$
\item We compute the new cost function, if this is less than the previous step, we keep the exchange, otherwise we revert the coordinate swap
\item We repeat steps 3 and 4 until the relative cost function change is less than a chosen accuracy parameter $\epsilon$ 
\end{enumerate}
%
We found that for $N=91$ grid points, order of $10^5$ iterations are sufficient to reach an accuracy of $\epsilon\sim10^{-4}$; once the optimal design $\mathcal{D}_m$ has been found, we can invert the mapping $\Pi\rightarrow[0,1]^3$ to find our simulation parameter sampling $\mathbf{p}$, which we show in 
Figure \ref{designfig}.
%
\begin{figure*}
\begin{center}
\includegraphics[scale=0.4]{Figures/design.eps}
\caption{$(\Omega_m,w)$ and $(w,\sigma_8)$ projections of our the simulation design; the blue points correspond to the CFHTemu1 simulation set, which consists of one $N$--body simulation per point, while the red point corresponds to the CFHTcov simulation set, which is based on 50 independent $N$--body simulations}
\label{designfig}
\end{center}
\end{figure*}
%
For each parameter point on the grid $\mathbf{p}$ we run one $N$--body simulation and perform ray tracing through it, as described in \S~\ref{raysim}, to simulate CFHTLens shear catalogs; this set of simulations will be called CFHTemu1 throughout the rest of this work. Additionally, we run 50 independent $N$--body simulations with a \textit{fiducial} parameter choice $\mathbf{p}_0=(0.26,-1.0,0.8)$ for the purpose of measuring accurately the covariance matrices which will serve the parameter inferences as in \S-\ref{cosmostats}; this set of simulations will be called CFHTcov throughout the rest of this work.    
\subsection{Ray Tracing Simulations}
\label{raysim}

%%%%%%%%%%%%%%%%%%%%%%%%%% METHODS %%%%%%%%%%%%%%%%%%%%%%%%%%%%%%%%%%%%%%%%%%%%%%%%%%%%%%

\section{Statistical methods}
The goal of this section is to describe the framework in which we combine the CFHT data and our simulations in order to derive the constraints on the cosmological parameters triplet $(\Omega_m,w,\sigma_8)$; we measure a set of statistical descriptors from the data and the simulations, which will then be compared in a bayesian fashion in order to compute parameter confidence intervals.

\subsection{Descriptors}
The statistical descriptors we consider on this work are the Minkowski Functionals (MFs) and the low order moments (LM) of the convergence field. The three MFs $(V_0,V_1,V_2)$ are topological descriptors of the convergence random field $\kappa(\pmb{\theta})$, which probe respectively the area, perimeter and genus characteristic of the $\kappa$ excursion sets $\Sigma_{\kappa_0}$, defined as $\Sigma_{\kappa_0}=\{\kappa>\kappa_0\}$. Following \citep{Petri2013,MinkJan} we use the following local estimators to measure the MFs from the $\kappa$ images
%
\begin{equation*}
\label{v0meas}
V_0(\kappa_0)=\frac{1}{A}\int_A\Theta(\kappa(\pmb{\theta})-\kappa_0)d\pmb{\theta},
\end{equation*}
\begin{equation}
\label{v1meas}
V_1(\kappa_0)=\frac{1}{4A}\int_A\delta(\kappa(\pmb{\theta})-\kappa_0)\sqrt{\kappa_x^2+\kappa_y^2}d\pmb{\theta},
\end{equation}
\begin{equation*}
\label{v2meas}
V_2(\kappa_0)=\frac{1}{2\pi A}\int_A\delta(\kappa(\pmb{\theta})-\kappa_0)\frac{2\kappa_x\kappa_y\kappa_{xy}-\kappa_x^2\kappa_{yy}-\kappa_y^2\kappa_{xx}}{\kappa_x^2+\kappa_y^2}d\pmb{\theta}.
\end{equation*}
%
Where $A$ is the total area of the fields of view we have been using and the notation $\kappa_{x,y}$ indicates gradients of the $\kappa$ field, which we evaluate using finite differences. The first Minkowski functional, $V_0$, is equivalent to the cumulative one--point PDF of the $\kappa$ field, while $V_1,V_2$ are sensitive to the correlations between nearby pixels. In addition to these topological descriptors, we consider a set of low order moments (LM) of the convergence field (two quadratic, three cubic and four quartic), which are defined in the following way
%
\begin{equation}
\begin{matrix}
\mathrm{LM_2}: \sigma_{0,1}^2 = \langle\kappa^2\rangle,\langle\vert\nabla\kappa\vert^2\rangle, \\ \\
\mathrm{LM_3}: S_{0,1,2} = \langle\kappa^3\rangle,\langle\kappa\vert\nabla\kappa\vert^2\rangle,\langle\kappa^2\nabla^2\kappa\rangle, \\ \\
\mathrm{LM_4}: K_{0,1,2,3} = \langle\kappa^4\rangle,\langle\kappa^2\vert\nabla\kappa\vert^2\rangle,\langle\kappa^3\nabla^2\kappa\rangle,\langle\vert\nabla\kappa\vert^4\rangle.
\end{matrix}
\end{equation}
%
If the $\kappa$ field was Gaussian, one could express all the Minkowski Functionals in terms of the LM2 moments, as expected from the fact that, for a Gaussian field, the only meaningful statistics are the quadratic ones. In reality, weak lensing convergence fields, as the CFHTLens ones, are non--Gaussian and the MF and LM descriptors are in principle equivalent. \citep{Munshi12,Matsubara10} studied a perturbative expansion of the MF descriptors in powers of the field standard deviation $\sigma_0$, which, when truncated at order $O(\sigma_0^2)$, can be expressed completely in terms of the LM up to quartic order. Such perturbative series, however, have been shown not to converge in \citep{Petri2013} unless the weak lensing fields are smoothed with windows of size comparable to $\sim 15^\prime$. Because of this, throughout this work, we treat MF and LM as independent statistical descriptors. 

\subsection{Cosmological parameter inferences}
\label{cosmostats}
In this paragraph we give a brief outline of the statistical framework we adopted for computing the cosmological parameter confidence levels from the CFHTLens observations; we make use of the MF and LM statistical descriptors outlined in the previous paragraph. We refer to $M_i^r(\mathbf{p})$ as the measured descriptor from a realization $r$ of one of our simulations with a choice of cosmological parameters $\mathbf{p}$, and to $D_i$ as the measured descriptor from the CFHTLens data. In this notation, $i$ is an index that refers to the particular bin on which the descriptor is evaluated (for example $i$ can range from 0 to 9 for the LM statistic and from 0 to $N_b-1$ for a Minkowski Functional measured on $N_b$ different excursion sets). Once we make an assumption for the data likelihood $\mathcal{L}_d(D_i\vert \mathbf{p})$ and for the parameter priors $\Pi(\mathbf{p})$, we can use the Bayes theorem to compute the parameter likelihood $\mathcal{L}_p$ as follows

\begin{equation}
\label{parameterlikelihood}
\mathcal{L}_p(\mathbf{p}\vert D_i) = \frac{\mathcal{L}_d(D_i\vert \mathbf{p})\Pi(\mathbf{p})}{N_{\mathcal{L}}}
\end{equation}
%
where $N_{\mathcal{L}}$ is a $\mathbf{p}$--independent constant that ensures the proper normalization for $\mathcal{L}_p$; we make the usual assumption that the data likelihood $\mathcal{L}_d(D_i\vert \mathbf{p})$ is a Gaussian

\begin{equation}
\label{datalikelihood}
\begin{matrix}
\mathcal{L}_d(D_i\vert \mathbf{p}) = ((2\pi)^{N_b}\det{\mathbf{C}})^{-1/2} e^{-\frac{1}{2}\chi^2(D_i\vert \mathbf{p})} \\ \\
\chi^2(D_i\vert \mathbf{p}) = \mathbf{(D - M(p))C^{-1}(D-M(p))}
\end{matrix}
\end{equation} 
%
The simulated descriptors $\mathbf{M(p)}$ are measured from an average over realizations
\begin{equation}
M_i(\mathbf{p}) = \frac{1}{R}\sum_{r=1}^R M_i^r 
\end{equation}
%
and the covariance matrix 
\begin{equation}
C_{ij} = \frac{1}{R-1} \sum_{r=1}^R [M_i^r(\mathbf{p}_0)-M_i(\mathbf{p}_0)][M_j^r(\mathbf{p}_0)-M_j(\mathbf{p}_0)]
\end{equation}
%
is measured from a simulation set based on 50 independent $N$--body simulations with parameters $\mathbf{p}_0=(0.26,-1.0,0.8)$ and is assumed to be model--independent; because of this the normalization constant in (\ref{datalikelihood}) is also model--independent. When computing parameter constraints from CFHTLens weak lensing data alone, we make a flat prior assumption for $\Pi(\mathbf{p})$. Parameter inferences are made estimating the location of the maximum of the parameter likelihood in (\ref{parameterlikelihood}), which we call $\mathbf{p}_{ML}(D_i)$ as well as its confidence contours. A $N\sigma$--confidence contour of $\mathcal{L}_p(\mathbf{p}\vert D_i)$ is defined to be the subset of points in parameter space on which the likelihood has a constant value $c_N$ and 
\begin{equation}
\int_{\mathcal{L}>c_N} \mathcal{L}_p(\mathbf{p}\vert D_i) d\mathbf{p} = \frac{1}{\sqrt{2\pi}}\int_{-N}^N dx e^{-x^2/2}
\end{equation}
%
Given the low dimensionality of the parameter space we consider $(N_p=3)$ we are able to evaluate the parameter likelihood (\ref{parameterlikelihood}) on a finely spaced $100\times100\times100$ mesh within the prior window $\Pi(\mathbf{p})$, and hence we are able to evaluate the likelihood maximum $\mathbf{p}_{ML}(D_i)$ and the contour levels $c_N$ directly without the need to use more sophisticated MCMC methods.

%%%%%%%%%%%%%%%%%%%%%%%%%% RESULTS %%%%%%%%%%%%%%%%%%%%%%%%%%%%%%%%%%%%%%%%%%%%%%%%%%%%%%

\section{Results}

%%%%%%%%%%%%%%%%%%%%%%%%%% DISCUSSION %%%%%%%%%%%%%%%%%%%%%%%%%%%%%%%%%%%%%%%%%%%%%%%%%%%%%%

\section{Discussion}

%%%%%%%%%%%%%%%%%%%%%%%%%% CONCLUSIONS %%%%%%%%%%%%%%%%%%%%%%%%%%%%%%%%%%%%%%%%%%%%%%%%%%%%%%

\section{Conclusions}

%%%%%%%%%%%%%%%%%%%%%%%%%% ACKNOWLEDGMENTS %%%%%%%%%%%%%%%%%%%%%%%%%%%%%%%%%%%%%%%%%%%%%%%%%%%%%%
 

\section*{Acknowledgements}
We thank the Stampede cluster who saved our day

\bibliography{ref}
\label{lastpage}
\end{document}
