%\documentclass[useAMS, usenatbib,usegraphicx,letter]{mn2e}
%\documentclass[11pt]{article}
\documentclass[reprint,aps,prd,superscriptaddress,showkeys,showpacs]{revtex4-1}
\usepackage{epsfig,amsmath,natbib}

\usepackage{aas_macros}
\usepackage{amssymb}
\usepackage{amsmath}
\usepackage{dsfont}
\usepackage{hyperref}
\usepackage{color}
\usepackage{pbox}

\hypersetup{
	colorlinks=false,
	citecolor=green
}
% \usepackage{graphicx}
% \usepackage{epstopdf}
% \usepackage{natbib}

\begin{document}

\title{CFHTLenS Weak Lensing Emulator and Cosmological Constraints from the Minkowski Functionals and Moments}

\author{Andrea Petri}
\email{apetri@phys.columbia.edu}
\affiliation{Department of Physics, Columbia University, New York, NY 10027, USA}
\affiliation{Physics Department, Brookhaven National Laboratory, Upton, NY 11973, USA}

\author{Jia Liu}
\affiliation{Department of Astronomy, Columbia University, New York, NY 10027, USA}

\author{Zolt\'an Haiman}
\affiliation{Department of Astronomy, Columbia University, New York, NY 10027, USA}

\author{Lam Hui}
\affiliation{Department of Physics, Columbia University, New York, NY 10027, USA}

\author{Jan M. Kratochvil}
\affiliation{Astrophysics and Cosmology Research Unit, University of KwaZulu-Natal, Westville, Durban 4000, South Africa}

\author{Morgan May}
\affiliation{Physics Department, Brookhaven National Laboratory, Upton, NY 11973, USA}

\date{\today}

\label{firstpage}

\begin{abstract}
Weak gravitational lensing has been proposed as an interesting probe for cosmology, and in particular for the Dark Energy (DE) equation of state $w$; the fact that typical weak lensing fields are non--Gaussian in nature, quadratic statistics, such as angular power spectra, can miss some of the cosmological information contained in survey data. In this work we examine constraints on the parameter triplet $(\Omega_m,w,\sigma_8)$ from weak lensing Minkowski Functionals and Moments, using the publicly available data from the 154\,deg$^2$ CFHTLenS survey. We utilize a new suite of ray--tracing N-body simulations spanning 91 points in the $(\Omega_m,w,\sigma_8)$ parameter space, replicating the galaxy sky positions, redshift and shape noise read from the CFHTLenS catalogs. We then build an emulator that interpolates the feature space with an accuracy of $\sim5\%$, and use it to compute the parameter likelihood, from which we derive the data constraints. We find that...

\end{abstract}

\keywords{Weak Gravitational Lensing --- Data analysis --- Methods: analytical,numerical,statistical}
\pacs{98.80.-k, 95.36.+x, 95.30.Sf, 98.62.Sb}

\maketitle


%%%%%%%%%%%%%%%%%%%%%%%%%% INTRO %%%%%%%%%%%%%%%%%%%%%%%%%%%%%%%%%%%%%%%%%%%%%%%%%%%%%%%%

\section{Introduction}
In this work we use the publicly available CFHTLenS data, that consists of a catalog of $\approx$4.2 million galaxies, combined with a suite of ray-tracing simulations in 91 different cosmological models to constrain the cosmological parameters $\Omega_m$,$\sigma_8$ and the DE equation of state $w$. The paper is organized as follows; we first give an overview of the CFHTLenS catalogs we make use of, summarizing the data reduction techniques we adopted. Next, we give a description of our simulation pipeline, including the procedure to sample the parameter space and a broad description of the ray--tracing algorithm; after this we illustrate the statistical descriptors that we use to extract the cosmological information, complemented with a dimensional reduction and statistical inference framework. We then outline our results, with particular focus on the cosmological parameter constraints. To conclude, we discuss our findings and comment on possible future developments of this analysis.  

%%%%%%%%%%%%%%%%%%%%%%%%%% DATA AND SIMULATIONS %%%%%%%%%%%%%%%%%%%%%%%%%%%%%%%%%%%%%%%%%%%%%%%%%%

\section{Data and simulations}

\subsection{CFHTLenS data reduction}
\label{cfhtdatareduction}
%
The CFHTLenS survey covers four sky patches of 64,23,44 and 23 deg$^2$ area, for a total of 154 deg$^2$; the publicly released data roughly consist of the creation of a galaxy catalogue using SExtractor \citep{SExtractor}, a photometric redshift estimation with a Bayesian photometric redshift code \citep{PhotoCode} and galaxy shape measurements with \textit{lensfit} \citep{cfht1,cfht2}. The cosmological parameter inferences have been obtained in \citep{CFHTKilbinger} using the two point correlation function (2PCF). We apply the following cuts to the galaxy catalogue: mask$<1$, redshift $0.2 < z < 1.3$, fitclass = 0 (which requires the object to be a galaxy) and weight $w>0$ (with larger $w$ indicating smaller shear measurement uncertainty). Applying these cuts leaves us 4.2$\times10^6$ galaxies, 124.7 deg$^2$ residual sky coverage, and average number density $n_{gal} \approx 9.3\,\mathrm{arcmin}^{-2}$. The CFHTLenS galaxy catalogue provides us with the sky position $\pmb{\theta}$, the redshift $z(\pmb{\theta})$ and ellipticity $\mathbf{e}(\pmb{\theta})$ of eack galaxy, as well as the individual weight factors $w(\pmb{\theta})$ and additive and multiplicative ellipticity corrections $c(\pmb{\theta}),m(\pmb{\theta})$. Because the CFHTLenS fields are irregularly shaped, we first divide them into 13 squares (subfields) to match the shape and $\approx12$ deg$^2$ size of our simulated maps; these square subfield maps are pixelized according to a Gaussian gridding procedure
\begin{equation}
\bar{\mathbf{e}}(\pmb{\theta}) = \frac{\sum_{i=1}^{N_s} W(\vert\pmb{\theta}-\pmb{\theta}_i\vert)w(\pmb{\theta}_i)[\mathbf{e}^{obs}(\pmb{\theta}_i)-c(\pmb{\theta}_i)]}{\sum_{i=1}^{N_s}W(\vert\pmb{\theta}-\pmb{\theta}_i\vert)w(\pmb{\theta}_i)[1+m(\pmb{\theta})]}
\end{equation} 
\begin{equation}
\label{gausskernel}
W_{\theta_G}(\pmb{\theta}) = \frac{1}{2\pi\theta_G^2}\exp{\left(-\frac{\pmb{\theta}^2}{2\theta_G^2}\right)}
\end{equation}
%
where the smoothing kernel $W_{\theta_G}$ has been varied choosing from the set $(0.5,1.0,1.8,3.5,5.3,8.9)\,$arcmin; the multiplicative and additive corrections $m,c$ related the observed and true ellipticities of the galaxies in the catalogue
\begin{equation}
\mathbf{e}^{obs} = (1+m)\mathbf{e}^{true} + c
\end{equation}
%
Using the ellipticity grid $\bar{\mathbf{e}}(\pmb{\theta})$ as an estimator for the cosmic shear $\gamma^{1,2}(\pmb{\theta})$, we can perform a non--local Kaiser--Squires inversion to recover the convergence $\kappa(\pmb{\theta})$ from the $E$--mode of the shear field
%
\begin{equation}
\kappa(\mathbf{l}) = \left(\frac{l_1^2-l_2^2}{l_1^2+l_2^2}\right)\gamma^1(\mathbf{l}) + 2\frac{l_1l_2}{l_1^2+l_2^2}\gamma^2(\mathbf{l})
\end{equation}
%
The CFHTLenS catalogues contain masked regions (mainly due to bright stars and incorrect PSF subtraction); these regions with low galaxy number density can induce large errors in the cosmological parameter inferences, hence they need to be masked out. We first create grided maps of the same size and resolution as the $\kappa$ maps, but with each pixel containing the number of galaxies ($n_{gal}$) falling within its window. We then smooth this galaxy surface density map with the same Gaussian window function as equation (\ref{gausskernel}) and we remove regions where $n_{gal} < 5 \,\mathrm{arcmin}^{−2}$ (see \citep{CFHTMasato}); for a more throughout description of our data reduction procedure, we refer the reader to our companion paper \citep{Companion}. 

\subsection{Simulation design}
In this paragraph we give a description of the method we used to sample the parameter space in our simulation effort. We wish to investigate the non--linear dependency of cosmological probes (in this work Minkowski Functionals and Moments of the $\kappa$ field) on the parameter triplet $\mathbf{p}=(\Omega_m,w,\sigma_8)$, while keeping fixed the other relevant parameters to $(h,\Omega_b,n_s)$ to their fiducial values (0.7,0.046,0.96). We sampled the $D$--dimensional ($D=3$ in this case) parameter space using an irregularly spaced grid, designed with a method similar to \citep{coyote2}. We limit the parameter sampling in a box of corners $\Omega_m\in[0.07,1],\,w\in[-3.0,0],\,\sigma_8\in[0.1,1.5]$ and we map this sampling box $\Pi$ into an hypercube of unit side; we want to contruct an irregularly spaced grid consisting of $N$ points $\mathbf{x}_i\in[0,1]^D$. Let a \textit{design} $\mathcal{D}$ be the set of this irregularly spaced $N$ points: we wish to find an optimal design, in which the points are spread as uniformly as possible inside the box. Following \citep{coyote2}, we choose our optimal design as the minimum of the cost function

\begin{equation}
\label{costfunction}
d(\mathcal{D}) = \frac{2D^{1/2}}{N(N-1)}\sum_{i<j}^N\frac{1}{\vert\mathbf{x}_i-\mathbf{x}_j\vert}
\end{equation} 
%
This problem is mathematically equivalent to the minimization of the Coulomb potential energy of $N$ unit charges in a unit box, which will make sure that the charges are as evenly spread as possible throughout the confining volume. Finding the optimal design $\mathcal{D}_m$ that minimized (\ref{costfunction}) can be computationally very demanding, and hence we decided to use a simplified approach that, although approximate, serves our purposes for the grid design. We use an iterative procedure:
\begin{enumerate}
\item We start from the diagonal design $\mathcal{D}_0$: $x_i^d\equiv i/(N-1)$
\item We shuffle the coordinates of the particles in each dimension independently $x_i^d = \mathcal{P}_d\left(\frac{1}{N-1},\frac{2}{N-1},...,1\right)$ where $\mathcal{P}_1,...,\mathcal{P}_D$ are random independent permutations of $(1,2,...,N)$
\item We pick a random particle pair $(i,j)$ and a random coordinate $d\in\{1,...,D\}$ and swap $x_i^d\leftrightarrow x_j^d$
\item We compute the new cost function, if this is less than the previous step, we keep the exchange, otherwise we revert the coordinate swap
\item We repeat steps 3 and 4 until the relative cost function change is less than a chosen accuracy parameter $\epsilon$ 
\end{enumerate}
%
We found that for $N=91$ grid points, order of $10^5$ iterations are sufficient to reach an accuracy of $\epsilon\sim10^{-4}$; once the optimal design $\mathcal{D}_m$ has been found, we can invert the mapping $\Pi\rightarrow[0,1]^3$ to find our simulation parameter sampling $\mathbf{p}_s$, which we show in Table \ref{designtable} and Figure \ref{designfig}.
%
\begin{table*}
\begin{tabular}{c|ccc||c|ccc||c|ccc||c|ccc}
$N$ & $\Omega_m$ & $w$ & $\sigma_8$ & $N$ & $\Omega_m$ & $w$ & $\sigma_8$ & $N$ & $\Omega_m$ & $w$ & $\sigma_8$ & $N$ & $\Omega_m$ & $w$ & $\sigma_8$ \\ \hline
1 & 0.136 & -2.484 & 1.034 & 26 & 0.380 & -2.424 & 0.199 & 51 & 0.615 & -1.668 & 0.185 & 76 & 0.849 & -0.183 & 0.821 \\
2 & 0.145 & -2.211 & 1.303 & 27 & 0.389 & -0.939 & 0.454 & 52 & 0.624 & -2.757 & 0.327 & 77 & 0.859 & -1.182 & 1.415 \\
3 & 0.155 & -0.393 & 0.652 & 28 & 0.399 & -1.938 & 1.500 & 53 & 0.634 & -1.575 & 0.976 & 78 & 0.869 & -2.031 & 0.227 \\
4 & 0.164 & -2.181 & 0.313 & 29 & 0.409 & -2.940 & 0.737 & 54 & 0.643 & -2.454 & 1.444 & 79 & 0.878 & -2.697 & 0.524 \\
5 & 0.173 & -0.423 & 1.231 & 30 & 0.418 & -1.758 & 0.383 & 55 & 0.652 & -1.029 & 1.458 & 80 & 0.887 & -0.363 & 0.439 \\
6 & 0.183 & -0.909 & 0.269 & 31 & 0.427 & -2.910 & 0.411 & 56 & 0.661 & -0.486 & 0.892 & 81 & 0.897 & -0.999 & 0.468 \\
7 & 0.192 & -1.605 & 1.401 & 32 & 0.436 & -0.060 & 0.878 & 57 & 0.671 & -2.364 & 0.793 & 82 & 0.906 & -1.698 & 1.273 \\
8 & 0.201 & -2.787 & 0.807 & 33 & 0.446 & -1.212 & 1.486 & 58 & 0.681 & -2.970 & 0.610 & 83 & 0.915 & -2.544 & 1.175 \\
9 & 0.211 & -0.333 & 0.341 & 34 & 0.455 & -2.637 & 1.373 & 59 & 0.690 & -1.332 & 0.482 & 84 & 0.925 & -0.636 & 1.259 \\
10 & 0.221 & -1.485 & 0.666 & 35 & 0.464 & -2.121 & 0.906 & 60 & 0.700 & -0.273 & 0.283 & 85 & 0.943 & -2.394 & 0.835 \\
11 & 0.239 & -1.848 & 0.962 & 36 & 0.474 & -1.302 & 0.114 & 61 & 0.709 & -2.061 & 0.425 & 86 & 0.953 & -1.545 & 0.355 \\
12 & 0.249 & -2.727 & 0.369 & 37 & 0.483 & -1.515 & 0.680 & 62 & 0.718 & -1.728 & 1.472 & 87 & 0.963 & -2.151 & 0.510 \\
13 & 0.258 & -1.395 & 0.241 & 38 & 0.493 & -0.243 & 0.297 & 63 & 0.728 & -0.120 & 0.596 & 88 & 0.972 & -0.666 & 0.694 \\
14 & 0.267 & -2.667 & 1.317 & 39 & 0.502 & -1.152 & 1.189 & 64 & 0.737 & -2.847 & 1.203 & 89 & 0.981 & -1.242 & 1.048 \\
15 & 0.276 & -0.849 & 1.429 & 40 & 0.512 & -0.819 & 0.849 & 65 & 0.746 & -0.090 & 1.118 & 90 & 0.991 & -1.908 & 1.020 \\
16 & 0.286 & -1.272 & 1.104 & 41 & 0.521 & -2.334 & 0.538 & 66 & 0.755 & -0.456 & 1.359 & 91 & 1.000 & -1.425 & 0.708 \\
17 & 0.295 & -1.878 & 0.100 & 42 & 0.530 & 0.000 & 0.624 & 67 & 0.765 & -2.091 & 1.076 & -- & -- & -- & -- \\
18 & 0.305 & -0.879 & 0.765 & 43 & 0.540 & -0.030 & 1.161 & 68 & 0.775 & -1.122 & 1.132 & -- & -- & -- & -- \\
19 & 0.315 & -2.241 & 0.638 & 44 & 0.549 & -1.818 & 1.287 & 69 & 0.784 & -1.062 & 0.779 & -- & -- & -- & -- \\
20 & 0.324 & -2.001 & 1.217 & 45 & 0.558 & -2.577 & 1.146 & 70 & 0.794 & -1.365 & 0.156 & -- & -- & -- & -- \\
21 & 0.333 & -0.213 & 0.552 & 46 & 0.568 & -0.516 & 1.331 & 71 & 0.803 & -2.607 & 0.255 & -- & -- & -- & -- \\
22 & 0.342 & -2.817 & 1.062 & 47 & 0.577 & -3.000 & 0.948 & 72 & 0.812 & -1.788 & 0.722 & -- & -- & -- & -- \\
23 & 0.352 & -0.576 & 1.090 & 48 & 0.587 & -2.304 & 0.128 & 73 & 0.821 & -2.880 & 0.863 & -- & -- & -- & -- \\
24 & 0.361 & -0.606 & 0.171 & 49 & 0.596 & -0.696 & 0.496 & 74 & 0.831 & -0.759 & 0.213 & -- & -- & -- & -- \\
25 & 0.370 & -0.303 & 1.345 & 50 & 0.606 & -0.789 & 0.142 & 75 & 0.840 & -2.274 & 1.387 & -- & -- & -- & -- \\
\end{tabular}
\caption{List of the CFHTemu1 grid points in parameter space}
\label{designtable}
\end{table*}
%
\begin{figure*}
\begin{center}
\includegraphics[scale=0.4]{Figures/design.eps}
\caption{$(\Omega_m,w)$ and $(\Omega_m,\sigma_8)$ projections of our the simulation design; the blue points correspond to the CFHTemu1 simulation set, which consists of one $N$--body simulation per point, while the red point corresponds to the CFHTcov simulation set, which is based on 50 independent $N$--body simulations}
\label{designfig}
\end{center}
\end{figure*}
%
For each parameter point on the grid $\mathbf{p}_s$ we run one $N$--body simulation and perform ray tracing through it, as described in \S~\ref{raysim}, to simulate CFHTLenS shear catalogs; this set of simulations will be called CFHTemu1 throughout the rest of this work. Additionally, we run 50 independent $N$--body simulations with a \textit{fiducial} parameter choice $\mathbf{p}_0=(0.26,-1.0,0.8)$ for the purpose of measuring accurately the covariance matrices which will serve the parameter inferences as in \S-\ref{cosmostats}; this set of simulations will be called CFHTcov throughout the rest of this work.    

\subsection{Ray Tracing Simulations}
\label{raysim}
The goal of this paragraph is to give an outline of our simulation pipeline; the fluctuations in the dark matter density field between a source at redshift $z$ and an observer located on Earth will cause small deflections to the trajectories of light rays travelling from the source to the observer. The fluctuations in the dark matter field are described by their gravitational potential $\Phi(\mathbf{x},z)=\Phi(\mathbf{x}_\perp,w(z))$, where we can trade the physical coordinates $\mathrm{x}$ with the comoving distance from the observer $w$ (not to be confused with the weight factor in \S\ref{cfhtdatareduction}) and two transverse coordinates $\mathbf{x}_\perp=w\pmb{\beta}$ using the flat sky approximation. Here $\pmb{\beta}$ refers to the angular coordinate on the sky of a physical point $\mathbf{x}$, as seen from the observer. We estimate the dark matter gravitational potential running $N$--body simulations (with $N=512^3$) with the public code Gadget2 \citep{Gadget2}, using a comoving box size of $240h^{-1}$Mpc. Using a similar procedure as in \citep{RayTracingJain,RayTracingHartlap}, the equation that governs the light ray deflections can be written in the form
\begin{equation}
\label{raytrajectory}
\frac{d^2\mathbf{x}(w)}{dw^2} = -\frac{2}{c^2}\nabla_{\mathbf{x}_\perp}\Phi(\mathbf{x}_\perp(w),w)
\end{equation}
%
where $\mathbf{x}(w)$ is the trajectory of a single light ray, which we visualize schematically in Figure \ref{rayscheme}. 
%
\begin{figure*}
\begin{center}
\includegraphics[scale=0.4]{Figures/rayscheme.eps}
\end{center}
\caption{Schematics of ray tracing with the lensing potential boosted by a factor of 50 for visualization purposes}
\label{rayscheme}
\end{figure*}
%
Suppose that a light ray reaches the observer at an angular position $\pmb{\theta}$ on the sky: we want to know where this light ray originated, knowing it comes from a redshift $z_s$. To answer this question we need to integrate equation (\ref{raytrajectory}) with the initial conditions $\pmb{\beta}(0;\pmb{\theta})=\pmb{\theta}$, $\dot{\pmb{\beta}}(0;\pmb{\theta})=0$ up to a distance $w_s=w(z_s)$ to obtain the source angular position $\pmb{\beta}(w_s;\pmb{\theta})$; for the light ray trajectory solver, based on equation (\ref{raytrajectory}), we use our proprietary implementation Inspector Gadget. Once we know the details of the light ray trajectories, we can easily infer the weak lensing interesting quantities by taking the angular derivatives of the ray deflections $A(w_s;\pmb{\theta}) = \partial \pmb{\beta}(w_s;\pmb{\theta})/\partial\pmb{\theta}$ and performing the usual spin decomposition to infer the convergence $\kappa$ and the shear components $(\gamma^1,\gamma^2)$
%
\begin{equation}
A(w_s;\pmb{\theta}) = (1-\kappa(w_s;\pmb{\theta}))\pmb{I} - \gamma^1(w_s;\pmb{\theta})\sigma^3 - \gamma^2(w_s;\pmb{\theta})\sigma^1
\end{equation}  
%
where $\pmb{I}$ is the $2\times2$ identity and $\sigma^{1,3}$ are the first and third Pauli matrices. $\kappa$ is related to the source apparent magnification, while $(\gamma^1,\gamma^2)$ are related to the source apparent ellipticity, as seen from the observer. Given a source with intrinsic ellipticity $\mathbf{e}_s=e^1_s + ie^2_s$, its observed ellipticity as seen from an observer will be modified by the cosmic shear $\pmb{\gamma}=\gamma^1 + i\gamma^2$ following
%
\begin{equation}
\mathbf{e} = 
\begin{cases}
\frac{\mathbf{e}_s+\mathbf{g}}{1+\mathbf{g}^*\mathbf{e}_s} \,\,\,\,\,\,\,\, \vert \mathbf{g}\vert \leq 1 \\ \\
\frac{1+\mathbf{ge}_s^*}{\mathbf{e}_s^* + \mathbf{g}^*} \,\,\,\,\,\,\,\, \vert \mathbf{g}\vert > 1
\end{cases}
\end{equation}
%
where $\mathbf{g} = \pmb{\gamma}/1-\kappa$ is the reduced shear. For each simulated galaxy, we assign an intrinsic elliptic- ity by rotating the observed ellipticity for that galaxy by a random angle on the sky, while conserving its magnitude $\vert\mathbf{e}\vert$. To be consistent with the CFHTLenS analysis, we adopt the weak lensing limit ($\vert\pmb{\gamma}\vert\ll1,\kappa\ll1$),hence $\mathbf{g}\approx\pmb{\gamma}$ and $\mathbf{e}\approx \mathbf{e}_s+\pmb{\gamma}$. We also add the multiplicative shear corrections by replacing $\pmb{\gamma}$ with $(1+m)\pmb{\gamma}$; like for the CFHTLenS data, we proceed in constructing the simulated $\kappa$ maps as explained in \S\ref{cfhtdatareduction}. These final simulation products are then processed together with the $\kappa$ maps obtained from the data to calculate the confidence intervals on the parameter triplet $(\Omega_m,w,\sigma_8)$.

%%%%%%%%%%%%%%%%%%%%%%%%%% METHODS %%%%%%%%%%%%%%%%%%%%%%%%%%%%%%%%%%%%%%%%%%%%%%%%%%%%%%

\section{Statistical methods}
The goal of this section is to describe the framework in which we combine the CFHT data and our simulations in order to derive the constraints on the cosmological parameter triplet $(\Omega_m,w,\sigma_8)$; we measure a set of statistical descriptors from the data and the simulations, which will then be compared in a bayesian fashion in order to compute parameter confidence intervals.

\subsection{Descriptors}
The statistical descriptors we consider on this work are the Minkowski Functionals (MFs) and the low order moments (LM) of the convergence field. The three MFs $(V_0,V_1,V_2)$ are topological descriptors of the convergence random field $\kappa(\pmb{\theta})$, which probe respectively the area, perimeter and genus characteristic of the $\kappa$ excursion sets $\Sigma_{\kappa_0}$, defined as $\Sigma_{\kappa_0}=\{\kappa>\kappa_0\}$. Following \citep{Petri2013,MinkJan} we use the following local estimators to measure the MFs from the $\kappa$ images. 
%
\begin{equation*}
\label{v0meas}
V_0(\kappa_0)=\frac{1}{A}\int_A\Theta(\kappa(\pmb{\theta})-\kappa_0)d\pmb{\theta},
\end{equation*}
\begin{equation}
\label{v1meas}
V_1(\kappa_0)=\frac{1}{4A}\int_A\delta(\kappa(\pmb{\theta})-\kappa_0)\sqrt{\kappa_x^2+\kappa_y^2}d\pmb{\theta},
\end{equation}
\begin{equation*}
\label{v2meas}
V_2(\kappa_0)=\frac{1}{2\pi A}\int_A\delta(\kappa(\pmb{\theta})-\kappa_0)\frac{2\kappa_x\kappa_y\kappa_{xy}-\kappa_x^2\kappa_{yy}-\kappa_y^2\kappa_{xx}}{\kappa_x^2+\kappa_y^2}d\pmb{\theta}.
\end{equation*}
%
Where $A$ is the total area of the fields of view we have been using and the notation $\kappa_{x,y}$ indicates gradients of the $\kappa$ field, which we evaluate using finite differences. The first Minkowski functional, $V_0$, is equivalent to the cumulative one--point PDF of the $\kappa$ field, $\partial V_0$ (which can be obtained by differentiation $\partial V_0(\kappa_0)=dV_0(\kappa_0)/d\kappa_0$), while $V_1,V_2$ are sensitive to the correlations between nearby pixels. In addition to these topological descriptors, we consider a set of low order moments (LM) of the convergence field (two quadratic, three cubic and four quartic), which are defined in the following way
%
\begin{equation}
\begin{matrix}
\mathrm{LM_2}: \sigma_{0,1}^2 = \langle\kappa^2\rangle,\langle\vert\nabla\kappa\vert^2\rangle, \\ \\
\mathrm{LM_3}: S_{0,1,2} = \langle\kappa^3\rangle,\langle\kappa\vert\nabla\kappa\vert^2\rangle,\langle\kappa^2\nabla^2\kappa\rangle, \\ \\
\mathrm{LM_4}: K_{0,1,2,3} = \langle\kappa^4\rangle,\langle\kappa^2\vert\nabla\kappa\vert^2\rangle,\langle\kappa^3\nabla^2\kappa\rangle,\langle\vert\nabla\kappa\vert^4\rangle.
\end{matrix}
\end{equation}
%
If the $\kappa$ field was Gaussian, one could express all the Minkowski Functionals in terms of the LM2 moments, as expected from the fact that, for a Gaussian field, the only meaningful statistics are the quadratic ones. In reality, weak lensing convergence fields, as the CFHTLenS ones, are non--Gaussian and the MF and LM descriptors are in principle equivalent. \citep{Munshi12,Matsubara10} studied a perturbative expansion of the MF descriptors in powers of the field standard deviation $\sigma_0$, which, when truncated at order $O(\sigma_0^2)$, can be expressed completely in terms of the LM up to quartic order. Such perturbative series, however, have been shown not to converge in \citep{Petri2013} unless the weak lensing fields are smoothed with windows of size comparable to $\sim 15^\prime$. Because of this, throughout this work, we treat MF and LM as independent statistical descriptors. In addition to these two descriptors, we consider the $\kappa$ angular power spectrum $P_l$ defined as
\begin{equation}
\label{powerspectrum}
\langle\tilde{\kappa}(\mathbf{l})\tilde{\kappa}(\mathbf{l}')\rangle=(2\pi)^2\delta_D(\mathbf{l}+\mathbf{l}')P_l
\end{equation}  
%
where $\tilde{\kappa}(\mathbf{l})$ is the Fourier transform of the $\kappa$ field and $\delta_D$ is the usual Dirac delta function; while the use of this statistic is not a novelty, we want indeed to compare the results we get using MF and LM to the ones already present in the literature. A summary of the statistical descriptors we used is presented in Table \ref{desctable}
%
\begin{table}
\begin{tabular}{c|c|c} \hline
Descriptor & Details & $N_b$ \\ \hline
$V_0,V_1,V_2$ (MF) & $\kappa_0\in[-0.04,0.12]$ & 50 \\
Power Spectrum (PS) & $l \in [300,5000]$ & 50 \\
Moments (LM) & -- & 9 \\
\end{tabular}
\caption{Summary of the descriptors we used, along with the specifications and the number of bins $N_b$ used}
\label{desctable}
\end{table}
%
\subsection{Cosmological parameter inferences}
\label{cosmostats}
In this paragraph we give a brief outline of the statistical framework we adopted for computing the cosmological parameter confidence levels from the CFHTLenS observations; we make use of the MF and LM statistical descriptors outlined in the previous paragraph. We refer to $M_i^r(\mathbf{p})$ as the measured descriptor from a realization $r$ of one of our simulations with a choice of cosmological parameters $\mathbf{p}$, and to $D_i$ as the measured descriptor from the CFHTLenS data. In this notation, $i$ is an index that refers to the particular bin on which the descriptor is evaluated (for example $i$ can range from 0 to 9 for the LM statistic and from 0 to $N_b-1$ for a Minkowski Functional measured on $N_b$ different excursion sets). Once we make an assumption for the data likelihood $\mathcal{L}_d(D_i\vert \mathbf{p})$ and for the parameter priors $\Pi(\mathbf{p})$, we can use the Bayes theorem to compute the parameter likelihood $\mathcal{L}_p$ as follows

\begin{equation}
\label{parameterlikelihood}
\mathcal{L}_p(\mathbf{p}\vert D_i) = \frac{\mathcal{L}_d(D_i\vert \mathbf{p})\Pi(\mathbf{p})}{N_{\mathcal{L}}}
\end{equation}
%
where $N_{\mathcal{L}}$ is a $\mathbf{p}$--independent constant that ensures the proper normalization for $\mathcal{L}_p$; we make the usual assumption that the data likelihood $\mathcal{L}_d(D_i\vert \mathbf{p})$ is a Gaussian

\begin{equation}
\label{datalikelihood}
\begin{matrix}
\mathcal{L}_d(D_i\vert \mathbf{p}) = ((2\pi)^{N_b}\det{\mathbf{C}})^{-1/2} e^{-\frac{1}{2}\chi^2(D_i\vert \mathbf{p})} \\ \\
\chi^2(D_i\vert \mathbf{p}) = \mathbf{(D - M(p))C^{-1}(D-M(p))}
\end{matrix}
\end{equation} 
%
The simulated descriptors $\mathbf{M(p)}$ are measured from an average over realizations
\begin{equation}
M_i(\mathbf{p}) = \frac{1}{R}\sum_{r=1}^R M_i^r 
\end{equation}
%
and the covariance matrix 
\begin{equation}
\label{datacov}
C_{ij} = \frac{1}{R-1} \sum_{r=1}^R [M_i^r(\mathbf{p}_0)-M_i(\mathbf{p}_0)][M_j^r(\mathbf{p}_0)-M_j(\mathbf{p}_0)]
\end{equation}
%
is measured from a simulation set based on 50 independent $N$--body simulations with parameters $\mathbf{p}_0=(0.26,-1.0,0.8)$ and is assumed to be model--independent; because of this the normalization constant in (\ref{datalikelihood}) is also model--independent. When computing parameter constraints from CFHTLenS weak lensing data alone, we make a flat prior assumption for $\Pi(\mathbf{p})$. Parameter inferences are made estimating the location of the maximum of the parameter likelihood in (\ref{parameterlikelihood}), which we call $\mathbf{p}_{ML}(D_i)$ as well as its confidence contours. A $N\sigma$--confidence contour of $\mathcal{L}_p(\mathbf{p}\vert D_i)$ is defined to be the subset of points in parameter space on which the likelihood has a constant value $c_N$ and 
\begin{equation}
\label{ennesigma}
\int_{\mathcal{L}>c_N} \mathcal{L}_p(\mathbf{p}\vert D_i) d\mathbf{p} = \frac{1}{\sqrt{2\pi}}\int_{-N}^N dx e^{-x^2/2}
\end{equation}
%
Given the low dimensionality of the parameter space we consider $(N_p=3)$ we are able to evaluate the parameter likelihood (\ref{parameterlikelihood}) on a finely spaced $100\times100\times100$ mesh within the prior window $\Pi(\mathbf{p})$, and hence we are able to evaluate the likelihood maximum $\mathbf{p}_{ML}(D_i)$ and the contour levels $c_N$ directly without the need to use more sophisticated MCMC methods. We know how to evaluate the data likelihood (\ref{datalikelihood}) on the simulation grid $\mathbf{p}_s$, but more work needs to be done to interpolate $M_(\mathbf{p})$ at an arbitrary intermediate point; we decided to use a Radial Basis Function (RBF) interpolation scheme. We approximate the model descriptor as
\begin{equation}
M(\mathbf{p}) = \sum_{s=1}^N \lambda_s\phi(\vert\mathbf{p}-\mathbf{p}_s\vert)
\end{equation}
%
where $\phi$ has been chosen as a multiquadric function $\phi(r)=\sqrt{1+(r/r_0)^2}$ with $r_0$ chosen as the mean euclidean distance between the points in the simulated grid $\mathbf{p}_s$. The constant coefficients $\lambda_s$ can be determined imposing the $N$ constraints $M(\mathbf{p}=\mathbf{p}_s)=M(\mathbf{p_s})$, which enforce the fact that the interpolation should be exact at the simulated points. The interpolation computations are conveniently performed using the Scipy library \citep{scipy} 

\subsection{Dimensionality reduction}
\label{pcasection}
%
The main goal of this work is constraining the cosmological parameter triplet $(\Omega_m,w,\sigma_8)$ using the CFHTLenS data; when the question on the accuracy of the $N\sigma$ contours is raised, particular attention should be given on the effect of binning choices on contour sizes computed with equations (\ref{parameterlikelihood})--(\ref{ennesigma}). It is known that the choice of the number of bins, $N_b$, can have a non--negligible effect on the contour sizes (see \citep{Petri2013} for an example with simulated datasets); in order for our results to be robust under this effect, we adopt a Principal Component Analysis (PCA) approach. Our main physical motivation for this approach is that, albeit we need to specify $N_b$ numbers in order to fully characterize a binned descriptor, we know that the effective dimensionality of the feature space should be close to the number of parameters that are effectively varied (see \citep{coyote2}), 3 in our case; because of this, we believe that dimensionality reduction techniques such as PCA can help in delivering constraints that are independent on the number of bins $N_b$ that we choose originally. In order to compute the Principal Components of our feature space, we consider the CFHTemu1 simulations, which sample the cosmological parameter space at the $N=91$ points outlined in Table \ref{designtable} and allow us to compute the $N\times N_b$ model matrix $M_{pi}=M_i(\mathbf{p})$. Following a standard procedure (see \citep{astroMLText} for example), we then derive the whitened model matrix $\tilde{M}_{pi}$ subtracting the mean in each bin, and normalizing it by its variance; next we proceed in a SVD decomposition of $\mathbf{\tilde{M}}$
\begin{equation}
\label{svd}
\mathbf{U}\Sigma \mathbf{V}^T=\frac{\mathbf{\tilde{M}}}{\sqrt{N-1}}
\end{equation}   
%
where $\Sigma_{ij}=\Sigma_i\delta_{ij}$ is a diagonal matrix and $V^T_{ij}$ is the $i$--th principal component of $\mathbf{\tilde{M}}$, with the index $j$ ranging from $0$ to $N_b-1$. To rank the Principal Components $V^T$ in order of importance, we note that the diagonal matrix $\Sigma^2$ is nothing more than the diagonalization of the model covariance (not to be confused with the feature covariance in (\ref{datacov}))
\begin{equation}
\frac{1}{N-1}\mathbf{\tilde{M}^T\tilde{M}} = \mathbf{V}\Sigma^2\mathbf{V}^T
\end{equation} 
%
We follow the standard interpretation of PCA components, stating that the only meaningful components $V^T_i$ in the analysis (i.e. the ones that contain the relevant cosmological information) are the ones that correspond to the biggest eigenvalues $\Sigma^2_{i}$, with the smallest eigenvalues corresponding to noisy patterns in the model, due to numerical inaccuracies in the simulation pipeline. We expect the number of relevant components to be not too far from 3; using the fact that different Principal Components are orthogonal, we perform a PCA projection on our feature space by whitening the features and computing the dot product with the principal components, keeping only the first $n$ components
\begin{equation}
\label{pcaprojection}
M(n)_{i}^r = V^T(n)_{ij}\tilde{M}_j^r \,\,\,\, ; \,\,\,\,  D(n)_i = V^T(n)_{ij}\tilde{D}_j
\end{equation}
%
where we indicate with $V^T(n)$ the truncation of $V^T$ to the first $n$ rows. The dimensional reduction comes from the fact that physical arguments justify a choice of $n< N_{b}$. We want to stress the fact that, because we vary only 3 parameters in the CFHTemu1 simulations, we expect our feature space to be a 3--dimensional manifold embedded in a $N_b$ dimensional space: the dimensionality reduction problem is equivalent to the accurate reconstruction of the coordinate chart of this feature manifold. As outlined in \citep{astroMLText}, the coordinate chart constructed with the PCA projection in (\ref{pcaprojection}) is accurate for reasonably flat feature manifolds: when curvature effects become important, more advanced projection techniques (such as Locally Linear Embedding) have to be employed. We are allowed some flexibility on this flatness assumption by allowing the number of principal components to be bigger than 3. 

%%%%%%%%%%%%%%%%%%%%%%%%%% RESULTS %%%%%%%%%%%%%%%%%%%%%%%%%%%%%%%%%%%%%%%%%%%%%%%%%%%%%%

\section{Results}
In this section we outline our main results

\subsection{Number of principal components}
%
The results of the PCA analysis are outlined in Figure \ref{pcafig}
\begin{figure*}
\includegraphics[scale=0.4]{Figures/pca_components.eps}
\caption{Principal Components of the $V_0$ (red), $V_1$ (blue), $V_2$ (green), LM (black) and Power Spectrum (orange) feature spaces; the left plot shows the magnitudes of the PCA eigenvalues $\Sigma_i^2$, the right plot shows their cumulative sum. A dashed black line has been drawn in correspondence of $n=3$ components}
\label{pcafig}
\end{figure*}

\subsection{Robustness}
%
The purpose of this paragraph is to convince the reader that the cosmological constraints we show in this paper are numerically robust, i.e. they are reasonably stable once we consider an increasing number of Principal Components. These stability plots are shown in Figure \ref{robustnessfig}

\begin{figure*}
\includegraphics[scale=0.4]{Figures/robustness_pca.eps}
\caption{Stability plot}
\label{robustnessfig}
\end{figure*}

\subsection{Cosmological constraints}

\begin{figure*}
\begin{center}
\includegraphics[scale=0.4]{Figures/contours_3comp.pdf}
\end{center}
\caption{$1\sigma$ constraints on the $(\Omega_m,\sigma_8)$ parameter doublet using the $V_0$ (red), $V_1$ (blue), $V_2$ (green), LM (black) and Power Spectrum (orange) statistics; the two panels are obtained marginalizing the parameter likelihood (\ref{parameterlikelihood}) over $w$ (left panel) and conditioning it on the plane where $w$ corresponds to its best fit value (right panel)}
\label{contours3comp}
\end{figure*}

\begin{table*}
\begin{tabular}{c|c|c|c|c||c}
Parameters & Descriptors & $\theta_G$ & Additional details & Short description & Relevant Figures/Tables \\ \hline \hline
$(\Omega_m,w,\sigma_8)$ & $V_0,V_1,V_2$, LM, PS & $1^\prime$ & BAD mask & $1\sigma$ contours from CFHTLenS data & Figure \ref{contours3comp} \\ \hline 
$(\Omega_m,w,\sigma_8)$ & $V_0,V_1,V_2$, LM, PS & $1^\prime$ & BAD mask &\pbox{20cm}{$1\sigma$ contours from simulations \\ for robustness test}  & Figure \ref{robustnessfig} \\
\end{tabular}
\caption{Summary table of our results}
\label{summarytable}
\end{table*}

\subsection{Combining statistics}

%%%%%%%%%%%%%%%%%%%%%%%%%% DISCUSSION %%%%%%%%%%%%%%%%%%%%%%%%%%%%%%%%%%%%%%%%%%%%%%%%%%%%%%

\section{Discussion}

%%%%%%%%%%%%%%%%%%%%%%%%%% CONCLUSIONS %%%%%%%%%%%%%%%%%%%%%%%%%%%%%%%%%%%%%%%%%%%%%%%%%%%%%%

\section{Conclusions}

%%%%%%%%%%%%%%%%%%%%%%%%%% ACKNOWLEDGMENTS %%%%%%%%%%%%%%%%%%%%%%%%%%%%%%%%%%%%%%%%%%%%%%%%%%%%%%
 

\section*{Acknowledgements}
We thank the Stampede cluster who saved our day

\bibliography{ref}
\label{lastpage}
\end{document}
