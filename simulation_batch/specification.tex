\documentclass[11pt]{article}

\usepackage{amssymb}
\usepackage{amsmath}
\usepackage{dsfont}
\usepackage{graphicx}
\usepackage{multirow}
\usepackage{hyperref}

\addtolength{\hoffset}{-2.5cm}
\addtolength{\textwidth}{4.5cm}
\addtolength{\voffset}{-2cm}
\addtolength{\textheight}{4.5cm}

\begin{document}

\title{N-body simulation batch for CFHTLens analysis}
\author{}
\date{}

\maketitle

\section*{Specifications}

We will run N-body simulations with Gadget2 according to the following specifications:

\begin{table}[h!]
\begin{center}
\begin{tabular}{ccc} \hline
Series name & mQ3 & \\
Number of particles & $512^3$ & \\ 
Number of snapshots per simulation & 60 & \\
Disk usage per simulation (including velocities) & 246\,GB& \\ \hline
\multicolumn{3}{c}{\textbf{Cosmological models to run}} \\
Id & $(\Omega_m,w,\sigma_8,h,\Omega_bh^2,n_s)$ & Number of simulations\\ \hline
Fiducial & (0.26,-1.0,0.8,0.72,0.0227,0.96) & 5 (+5 pending AP...) \\
Variation 1 & (0.26,-0.8,0.8,0.72,0.0227,0.96) & 5 \\
Variation 2 & (0.29,-1.0,0.8,0.72,0.0227,0.96) & 5 \\
Variation 3 & (0.26,-1.0,0.85,0.72,0.027,0.96) & 5 \\ \hline
Double variation 1 & (0.26,-0.6,0.8,0.72,0.0227,0.96) & 1 + 9 pending AP... \\
Double variation 2 & (0.32,-1.0,0.8,0.72,0.0227,0.96) & 1 + 9 pending AP...) \\
Double variation 3 & (0.26,-1.0,0.9,0.72,0.027,0.96) & 1 + 6 pending AP...) \\ \hline
\end{tabular}
\end{center}
\end{table}

\section*{Note on Gadget memory usage}
The size of our simulations is limited by the computing resources we can use on Blue Gene Q; with a 128 node partition at full usage, we are able to run 16 simulations of $512^3$ particles in parallel, which take order of 30 hours to complete. 

\end{document}
